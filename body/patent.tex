\chapter{答辩委员会决议}
自监督推荐充分利用输入数据中的内在关系提取自监督信号,有助于缓解数据稀疏、泛化误差等推荐领域问题,具有重要的理论意义和实用价值。论文面向自监督推荐领域,针对基于对比学习的推荐算法设计中的伪正例、伪负例和优化目标偏离问题,提出了正例去噪算法、负例采样算法和损失函数校正算法,主要贡献和创新点如下:

1.针对伪正样本问题,提出了自适应去噪的推荐算法,将从含噪成对比较的数据中学习排序的问题形式化为一个含有隐变量的最大后验估计问题,实验表明其在带有噪声的数据集上具有更好推荐准确性和鲁棒性。

2.针对伪负样本问题,提出了基于最优负采样准则的推荐算法,设计了基于后验概率的负信号测度,以及经验采样风险最小化的采样准则,该算法在采样误差率和推荐准确性上优于同类方法。

3.针对自监督场景下成对损失优化和对比损失优化目标偏离问题,提出了损失函数校正算法,设计了去偏成对损失,以及贝叶斯自监督对比损失,取得了更高的推荐准确性。 

论文结构合理,逻辑清晰,写作规范,实验充分。论文答辩过程中,表述清楚,能正确回答评委提出的问题。研究工作表明作者已掌握本学科坚实宽广的基础理论和系统深入的专门知识,具有独立从事科研工作能力。

答辩委员会经过无记名投票,认为本论文已达到了博士学位论文的水平,一致同意通过博士论文答辩,建议授予刘斌工学博士学位。

\begin{flushright} 
\textbf{答辩委员会主席: 刘文予}\\
答辩日期:2023年11月27日
\end{flushright}




