
%%% Local Variables:
%%% mode: latex
%%% TeX-master: "../main"
%%% End:

\begin{ack}

光阴荏苒,我踏入即将毕业的时刻,回首研究生岁月,这是一段难忘的旅程。这段路上,欢声笑语相伴,挑战与成长交织。衷心感谢那些在旅途中给予我帮助的人,向他们表达最真挚的谢意。

首先我要感谢我的导师王邦教授。感谢他对我的细心培养和教导。王老师见解独到而深刻,他思考的力度、广度和深度,永远是我学习的榜样。当我还在处理正例噪声问题时,王老师敏锐地意识到负采样问题的实质与我所熟悉的任务具有相似性,并指引我开始从事负采样的研究,才有后续的成果;王老师治学严谨,他对研究的每个细节都精益求精,学术探索中追求真理的决心始终如一。对于我的每一篇论文,从遣词造句到公式,王老师都亲力亲为从头到尾修改,仔细校对每句话、每个公式的正确性,在过年和生病期间也从未有例外。即使是\LaTeX~文档引用标签,王老师也会亲力亲为地按照驼峰命名规则进行修改。王老师的身先示范,让我学会了严谨治学,踏踏实实做好每一件事。王老师的帮助并不仅限于科研工作,他对我对未来的规划也倾注关怀。他积极主动地了解我的想法,并给予了宝贵的意见和帮助。非常感谢王老师的指导与帮助,师恩厚重。

其次我要感谢我的启蒙老师张宏志,没有张老师我不会走上学术道路。张老师循序善诱,总是能够深入浅出地讲述各种晦涩理论,他讲述的极大似然原理让我至今记忆犹新;张老师和蔼可亲,在数学竞赛近半年的培训期间,他发现我有问题不明白,只说“这里是我没讲好”,从不责怪我不用心。感谢张老师对我的引导及启蒙。

其次,我要感谢\textsc{mins}实验室一起学习和奋斗的兄弟姐妹们。感谢师兄邓贤君、刘生昊在学习和生活上的关心;感谢陈尔嘉、罗勤、张子卓、逯凌云、石昭、宋世澎、刘文轩、丁宇晗、杨雪娇、尚子桥的合作与帮助;感谢代璐、项威、王人玉、陈思、张家政、綦廷浩、祈吉明、曹虎朋、韦李潇、欧阳光、余涵、罗紫菱、王正林、刘松涛、刘程、梁超、占传鸿等小伙伴的相互激励与欢声笑语。

我还要感谢我的同学和朋友冉龙亚、郭文山、罗强强、孙宁华、袁瀚坤、贾鹏、陶相飞、马军舰;特别感谢雷军、万甲双、赵森森、张印在我困难和迷茫时期给予的无私帮助。


最后我要感谢我的父母,你们一直是我人生道路上最坚实的后盾。每当我做出决定时,你们总是无条件地付出、理解和支持,给予我勇气和信心去追求我的梦想。对于同龄人结婚生子的选择,你们的焦虑只是藏在心里,并未写在脸上。这种默默的关爱让我感到无比温暖,也使我更加坚定地前行。

在我攻读博士学位的旅程中,我经历了无数的挑战和困难。然而,每当我感到疲惫和迷茫时,你们总是在我身边给予我力量和勇气。再次感谢这一路上帮助过我和支持过我的人们。

你们的支持和鼓励是我孜孜以求的不竭动力。
\end{ack}
