\chapter{成对损失函数校正研究}
\label{cha:fourthsection}
从成对比较中学习对比表示(contrastive representations)在自然语言处理、计算机视觉和信息检索等多个领域取得了显著的成功。基于成对学习的协同过滤算法也根植于这一范式。一个重要的问题是数据集普遍是positive-unlabeled,即负例未标注,这经常导致随机选择的负例包含了伪负例,并且不可避免地引入导致学到有偏的嵌入,从而影响下游的分类、排序任务的泛化性能。上一章介绍了最优负采样准则,但是负采样算法不适用于基于GPU批量并行计算的mini-batch模型,往往以额外的存储开销和计算开销为代价实现。本章采用了一个截然不的技术路径纠正采样偏差,得到了一种校正后的成对学习损失函数,称为去偏成对损失(Debiased Pairwise Loss, DPL)。DPL的核心思想是纠正由于伪负例引起的偏倚概率估计,从而修正梯度,使其逼近完全监督数据的梯度。对五个公共数据集进行的实验研究验证了所提出的学习方法的有效性。

\section{引言}
成对学习鼓励编码器对样本之间的差异特征进行编码,而不是对个体样本的像素级特征进行编码,通常能够获得更好的泛化性能,尤其是在样本的绝对值较少有意义的场景下~\cite{Wang:2020:ICML,McFadden:1974:FE,gutmann:2012:JMLR,Liu:2021:TKDE,Wang:2020:ICML}。成对学习已成为许多现代机器学习算法的基本组成部分,并在自然语言处理、图像和语音识别以及推荐系统等诸多领域取得了重大进展~\cite{Oord:2018:arxiv,Wang:2020:ICML,He:2020:CVPR,Steffen:2009:UAI}。在协同过滤中,成对学习被广泛应用于预测排序,其中最著名的方法是贝叶斯个性化排名(BPR)~\cite{Steffen:2009:UAI}。从统计角度来看,BPR最大化了观察到的正负样本的后验概率。从数值计算角度来看,BPR损失函数鼓励模型正例的得分高于负例。在嵌入空间中,BPR损失的目标是将正例的嵌入拉近到锚点(即用户),同时将负例的嵌入推离锚点。

然而,用户通常只通过交互行为(如点击、购买或评分)来表达他们的偏好或兴趣,从而仅提供正反馈。因此,训练集通常以正无标签(PU)数据的形式存在。针对这一在机器学习的诸多领域普遍存在的问题,负采样技术被广泛研究。负采样可以分为两种类型:第一种是静态负采样,它使用固定的采样分布,利用与模型的训练状态无关的某种辅助信息,可能导致简单的样本,相比之下,与动态负采样相比,静态负采样的性能往往较差。此外,静态负采样严重依赖于辅助信息作为有效的监督信号,而这些监督信息在大多数场景下是难以获取的。动态负采样使用与模型相关的信息(如预测得分)动态调整采样分布,旨在采样具有高得分或排名靠前的困难负样本,以提高性能,但容易遇到伪负例~\cite{Ding:2020:NIPS}。此外,基于GPU批量计算的mini-batch训练与动态负采样不兼容。矛盾的关键在于,mini-batch训练要求训练之前将样本固定加载到数据加载器中,然后进行正向预测得分和反向传播更新参数;而动态负采样要求先正向预测得分,然后采样样本。动态负采样通常需要额外的计算和存储开销,例如存储先前训练轮次的预测得分~\cite{Ding:2020:NIPS},或计算mini-batch之外样本的预测得分。

本章聚焦于没有任何辅助信息用于监督的最一般形式的隐式反馈数据,提出了一种从无标签数据中纠正采样偏差的方法,从而为成对学习提供了一种修改后的损失函数,称为去偏成对损失(Debiased Pairwise Loss,DPL)。DPL的核心思想是修正由于假负例导致的概率估计偏差,从而修正梯度以近似完全监督数据的梯度。DPL不需要额外的辅助信息进行监督,也不需要过多的存储和计算开销。

\section{成对损失优化目标分析}
将用户物品对$(u,i)$表示为样本$\mathbf{x}$,其中$u\in \mathcal{U}$,$i\in \mathcal{I}$。令$\mathcal{X}= \{\mathbf x|u\in \mathcal{U}, i\in \mathcal{I}\}$表示包含所有用户物品对的样本空间,$\mathcal{Y} =\{-1,+1\}$表示类别标签,指示用户是否喜欢该物品。决策函数$g:\mathcal{X} \rightarrow \mathbb{R}$是一个实值函数,为每个交互分配一个预测的偏好水平$g(\mathbf x) \in \mathbb{R}$。将正例的类条件密度表示为$p^+(\mathbf x) = p(\mathbf x|+1)$,将负例的类条件密度表示为$p^-(\mathbf x) = p(\mathbf x|-1)$。因此,边际分布$p(\mathbf x)=p^+(\mathbf x) \tau^+ +p^-(\mathbf x)\tau^-$,其中$\tau^+ = 1-\tau^-$是先验概率$p(c(\mathbf{x}) = +1)$。

在隐式协同过滤中,通过对两个随机样本$(\mathbf{x}^+, \mathbf{x}^-)$进行成对比较,通过预测的偏好水平$g(\mathbf x) \in \mathbb{R}$以预测个性化排序。BPR\cite{Steffen:2009:UAI}采用了著名的Bradley-Terry模型建模观察到正例$\mathbf{x}^+$优于负例$\mathbf{x}^-$的似然
\begin{eqnarray}\label{eq:bpr}
	\mathcal{L}_{BPR} &=& - \mathbb{E}_{\substack{\mathbf x^+ \sim p^+(\mathbf x) \\ \mathbf x^- \sim p^-(\mathbf x^-)}} \log \sigma(g(\mathbf{x}^+) - g(\mathbf{x}^-)) \\
	&=&  - \mathbb{E}_{\substack{\mathbf x^+ \sim p^+(\mathbf x) \\ \mathbf x^- \sim p^-(\mathbf x)}}\log \frac{1}{1+\exp(-g(\mathbf{x}^+) + g(\mathbf{x}^-))} \nonumber \\
	&=&  - \mathbb{E}_{\substack{\mathbf x^+ \sim p^+(\mathbf x) \\ \mathbf x^- \sim p^-(\mathbf x^-)}}\log\frac{\exp(g(\mathbf{x}^+))}{\exp(g(\mathbf{x}^+))+\exp( g(\mathbf{x}^-))} \label{eq:infonce1}
\end{eqnarray}

值得注意的是,式\eqref{eq:infonce1}是贝叶斯个性化排名(BPR)损失的等价形式,它与噪声对比估计(NCE)损失\cite{Gutmann:2010:ICAIS}完全相同,而NCE损失是InfoNCE损失\cite{Oord:2018:arxiv}在只有一个负例(即N=1)的特殊情况。此外,在协同过滤的场景中,用户和物品构成一个二部图,通常只选择用户嵌入作为锚点。在理想情况下,正例是从已有交互的物品中进行采样的,表示为$\mathbf{x}^+ \in \mathcal{D}^+$,而负例是从用户不喜欢的物品中进行采样的,表示为$\mathbf{x}^- \in \mathcal{D}^-$。因此,式\eqref{eq:bpr}的对应的经验估计可以表示为:
\begin{eqnarray}\label{eq:bpr_emp}
	\mathcal{L}_\text{BPR} =- \frac{1}{|\mathcal{D}^+|\times |\mathcal{D}^-|} \sum_{\mathbf{x}^+ \in \mathcal{D}^+}\sum_{\mathbf{x}^- \in \mathcal{D}^-} && \ln \sigma(g(\mathbf{x}^+) - g(\mathbf{x}^-)) \nonumber  - \lambda ||\Theta||^2,
\end{eqnarray}
其中,$\lambda ||\Theta||^2$ 是正则化项,用于平衡方差和偏差,避免过拟合。特别地,正则化项$\lambda ||\Theta||^2$等价于高斯分布的先验密度的对数,从而为式\eqref{eq:bpr_emp}提供了基于后验概率的解释。在原始的贝叶斯个性化排名(BPR)论文\cite{Steffen:2009:UAI}中,式\eqref{eq:bpr_emp}被解释为观测到的有序对的最大后验估计。

在实际中,由于缺乏标记的负例样本,在优化式\eqref{eq:bpr_emp}时,只能从未标记的数据中采样负例样本,导致以下偏倚的优化目标:
\begin{eqnarray}\label{eq:bpr_emp_biased}
	\mathcal{L} =- \frac{1}{|\mathcal{D}^+|\times |\mathcal{D}^u|} \sum_{\mathbf{x}^+ \in \mathcal{D}^+}\sum_{\mathbf{x} \in \mathcal{D}^u} && \ln \sigma(g(\mathbf{x}^+) - g(\mathbf{x})) - \lambda ||\Theta||^2,
\end{eqnarray}

接下来,考察偏倚的优化目标对学习到的用户-物品嵌入$\Theta$的影响:
\begin{eqnarray}
	\frac{\partial \mathcal{L}}{\partial \Theta} &=& \frac{\partial \mathcal{L}_\text{BPR}}{g(\mathbf{x})}\cdot\frac{g(\mathbf{x})}{\Theta}  \\
	&=& [1-\sigma(g(\mathbf{x}^+) - g(\mathbf{x})) ]\cdot\frac{g(\mathbf{x})}{\Theta}  \label{eq:graident}
\end{eqnarray}
式\eqref{eq:graident}是微分链式法则的结果,其中第一项$[1-\sigma(g(\mathbf{x}^+) - g(\mathbf{x}))]$由损失函数的形式决定,第二项$\frac{g(\mathbf{x})}{\Theta}$由决策函数决定。对于固定的模型,第二项保持不变。

在式\eqref{eq:graident}的第一项中,实值sigmoid函数$\sigma(g(\mathbf{x}^+) - g(\mathbf{x})) \in [0,1]$被解释为用户偏好正例强于负例的似然\cite{Steffen:2009:UAI},是优化目标。然而,由于$\mathbf{x}$是未标记样本,它以$\tau^+$的先验概率是正样本,这导致了概率的偏倚估计。

图\ref{fig:event}提供了一个阐释性的例子说明为何使用正例-未标记数据对计算的值$\sigma(g(\mathbf{x}^+) - g(\mathbf{x}))$是一个偏倚的概率估计。“偏好正例强于负例的似然”,描述的是事件$\mathcal{A}_1(+,-)$的概率,这是优化目标。然而在实践中,给出的是正例-未标注样本对,导致成对学习优化的是事件$\mathcal{A}(+,u)$的概率。然而,事件$\mathcal{A}(+,u)$的概率和事件$\mathcal{A}_1(+,-)$的概率是不等价的,根据事件的可列可加性原理,事件$\mathcal{A}(+,u) = \mathcal{A}_1(+,-) + \mathcal{A}_2(+,+)$。这个偏差,导致了偏倚的“偏好正例强于负例”的概率估计,从而导致了有偏的最大似然/后验的优化目标。直观地说,偏倚的$\sigma(g(\mathbf{x}^+) - g(\mathbf{x}))$值会导致不正确的梯度大小${\partial \mathcal{L}_\text{BPR}}/{\partial \Theta}$,从而在执行随机梯度下降学习算法时导致不准确的用户-物品表示。
%%%%%%%%%%%%%%%%%%%%%%%%%%%%%%%%%%%%%%%%%%%%%%%%%%%%%%%%%%%
\begin{figure}[h!]
	\centering
	\includegraphics[width=0.8\textwidth]{4-event.png}
	\caption{偏倚的概率估计的示意图} 
	\label{fig:event}
\end{figure}
%%%%%%%%%%%%%%%%%%%%%%%%%%%%%%%%%%%%%%%%%%%%%%%%%%%%%%%%%%%

为了使数学表达中保持简洁,本章定义一个映射$h: \mathcal{X}\times\mathcal{X} \rightarrow \sigma(g(\mathbf{x}^+) - g(\mathbf{x}))$,将两个样本映射到一个sigmoid函数。问题变成如何使用来自正例和未标记样本群体的样本来近似估计$h(\mathbf{x}^+,\mathbf{x}^-)$的值。具体而言,给定一组正例样本$\{\mathbf{x}_i\}_{i=1}^M$和一组未标记样本$\{\mathbf{x}_j\}_{j=1}^N$,目标是估计$h(\mathbf{x}^+,\mathbf{x}^-)$的值,从而可以校正梯度以近似完全监督数据的梯度,从而提高学习到的用户-物品表示的泛化性能。

\section{去偏成对学习算法}
为了使用来自正例-未标记样本对来近似估计$h(\mathbf{x}^+,\mathbf{x}^-)$的值,首先建立正例-未标记样本对的联合分布$p_\textsc{pu}$与正例-负例样本对的联合分布$p_\textsc{pn}$之间的关系。

样本对的第一个样本,记为$\mathbf x_1$,是从正类条件概率$p^+(\mathbf x)$中确定性地抽取的,而第二个样本,记为$\mathbf{x}_2$,是从边缘分布$p(\mathbf{x})$中抽取的。因此,正例-未标记样本对的联合分布$p\textsc{pu}$可以表示如下:
\begin{eqnarray}
	p_\textsc{pu}(\mathbf{x}_1, \mathbf{x}_2) &=& p^+( \mathbf{x}_1) p( \mathbf{x}_2)  \label{eq:independent} \\
	&=& p^+( \mathbf{x}_1) [p^+(\mathbf x_2) \tau^+ +p^-(\mathbf x_2)\tau^- ] \label{eq:full}\\
	&=& \tau^+p^+( \mathbf{x}_1) p^+(\mathbf x_2) + \tau^-p^+( \mathbf{x}_1)p^-(\mathbf x_2) \label{eq:pnpp}
\end{eqnarray}
式\eqref{eq:independent}是由于$(\mathbf{x}_1, \mathbf{x}_2)$是独立抽取的。此外,式\eqref{eq:full}是边缘分布$p(\mathbf{x})$的完整概率分解。重新整理式\eqref{eq:pnpp},可以建立所需的正例-负例样本对的联合分布$p_\textsc{pn}(\mathbf{x}_1, \mathbf{x}_2)$与正例-未标记样本对的联合分布$p_\textsc{pu}(\mathbf{x}_1, \mathbf{x}_2)$之间的关系,表示为
\begin{eqnarray}\label{eq:jointpn}
	p_\textsc{pn}(\mathbf{x}_1, \mathbf{x}_2)  &=& p^+( \mathbf{x}_1)p^-(\mathbf x_2) \nonumber \\
	&=& \frac{1}{\tau^-}[p_{\textsc{pu}}(\mathbf{x}_1, \mathbf{x}_2)- \tau^+p^+( \mathbf{x}_1) p^+(\mathbf x_2)] \label{eq:pndist}
\end{eqnarray}

因此,$h(\mathbf x_1,\mathbf x_2)$相对正例-负例样本对的联合分布$p_\textsc{pn}(\mathbf{x}_1, \mathbf{x}_2)$上的期望,即理想的优化目标“用户喜欢正例$\mathbf{x}_1$优于负例$\mathbf{x}_2$”的概率期望值,可以计算如下:
\begin{eqnarray}
	&&\mathbb{E}_{ p_\textsc{pn}(\mathbf{x}_1, \mathbf{x}_2)} h(\mathbf{x}_1,\mathbf{x}_2)\\
	&=& \int_{\mathbf{x}_1}\int_{\mathbf{x}_2}   h(\mathbf{x}_1,\mathbf{x}_2) p_\textsc{pn}(\mathbf{x}_1, \mathbf{x}_2) d{\mathbf{x}_1}d{\mathbf{x}_2} \\
	&=&\int_{\mathbf{x}_1}\int_{\mathbf{x}_2}   h(\mathbf{x}_1,\mathbf{x}_2) [\frac{1}{\tau^-}p_{\textsc{pu}}(\mathbf{x}_1, \mathbf{x}_2)- \frac{\tau^+}{\tau^-}p^+( \mathbf{x}_1) p^+(\mathbf x_2)]d{\mathbf{x}_1}d{\mathbf{x}_2} \\
	&=&\int_{\mathbf{x}_1}\int_{\mathbf{x}_2}   h(\mathbf{x}_1,\mathbf{x}_2) [\frac{1}{\tau^-}p_{\textsc{pu}}(\mathbf{x}_1, \mathbf{x}_2) d{\mathbf{x}_1}d{\mathbf{x}_2}- \int_{\mathbf{x}_1}\int_{\mathbf{x}_2} \frac{\tau^+}{\tau^-}p^+( \mathbf{x}_1) p^+(\mathbf x_2)]d{\mathbf{x}_1}d{\mathbf{x}_2} \\
	&=& \frac{1}{\tau^-}\mathbb{E}_{ p_\textsc{pu}(\mathbf{x}_1, \mathbf{x}_2)} h(\mathbf{x}_1,\mathbf{x}_2) -\frac{\tau^+}{\tau^-} \mathbb{E}_{ p_\textsc{pp}(\mathbf{x}_1, \mathbf{x}_2)} h(\mathbf{x}_1,\mathbf{x}_2) \label{eq:expec}
\end{eqnarray}
式\eqref{eq:expec}的第一项表示正例-未标记样本对的期望,在给定第一个正例样本$\mathbf{x}^+$的情况下,可以使用$N$个未标记样本$\{\mathbf{x}_n\}_{n=1}^N$进行经验估计:
\begin{eqnarray}
	\mathbb{E}_{ p_\textsc{pu}(\mathbf{x}_1, \mathbf{x}_2)} h(\mathbf{x}_1,\mathbf{x}_2) &=& \frac{1}{N} \sum_{n=1}^{N} h(\mathbf{x}^+,\mathbf{x}_n)\label{eq:epu},
\end{eqnarray}
式\eqref{eq:expec}的第二项表示正例-正例样本对的期望,在给定第一个正例样本$\mathbf{x}^+$的情况下,可以使用额外的$M$个正例样本${\mathbf{x}'_n}_{m=1}^M$进行经验估计
\begin{eqnarray}
	\mathbb{E}_{ p_\textsc{pp}(\mathbf{x}_1, \mathbf{x}_2)} h(\mathbf{x}_1,\mathbf{x}_2)  &=& \frac{1}{M} \sum_{m=1}^{M} h(\mathbf{x}^+,\mathbf{x}_m^{\prime})\label{eq:epp}.
\end{eqnarray}

将式\eqref{eq:epu}和式\eqref{eq:epp}代入式\eqref{eq:expec},可以得到理想的优化目标“用户喜欢正例$\mathbf{x}_1$优于负例$\mathbf{x}_2$”的概率期望值的经验估计:
\begin{eqnarray}\label{eq:est}
	\hat{P}_\textsc{pn} = \frac{1}{N\tau^-}
	\sum_{\mathbf{x} \in \mathcal{D}^u}h(\mathbf{x}^+,\mathbf{x})  - \frac{\tau^+}{M\tau^-}\sum_{\mathbf{x}^\prime\in \mathcal{D}^+}h(\mathbf{x}^+, \mathbf{x}^\prime)
\end{eqnarray}

如上所示,方程中的第一项使用正例-未标记样本对进行计算,但额外添加了一个校正项,以抵消由于未标记集合中包含错误的负例样本而产生的偏倚似然估计。式\eqref{eq:est}中的第二项,从正例-正例对计算的校正项可能看起来奇怪,图\ref{fig:auc}对这个校正项提供了一个直观解释。

考虑一个正例-未标记的数据集,其中$\{x_1, x_2\}$是正例样本,$\{x_3, x_4, x_5\}$是未标记样本。在训练过程中,无法访问未标记样本$\{x_3, x_4, x_5\}$的真实标签。式\eqref{eq:est}的第一项描述的是使用正例样本和未标记数据对计算的AUC风险,即把不可微的0-1损失替换为可微的\textsf{sigmoid}损失所得到的AUC值,记为$AUC_\textsc{pu}$。根据AUC的定义,$AUC_\textsc{pu}$可以分解为两个项的和组成。第一个项是使用正例-负例(PN)数据对计算的干净AUC估计,表示为$AUC_\textsc{pn}$。第二个项是使用正例-正例(PP)数据对计算的伪AUC估计,表示为$AUC_\textsc{pp}$。为了优化干净AUC,应该从$AUC_\textsc{pu}$中减去$AUC_\textsc{pp}$。而减去的补偿项目,正是由式\eqref{eq:est}的第二项基于正例-正例对计算的伪AUC估计。

%%%%%%%%%%%%%%%%%%%%%%%%%%%%%%%%%%%%%%%%%%%%%%%%%%%%%%%%%%%
\begin{figure*}[h!]
	\centering
	\includegraphics[width=\textwidth]{4-unbiasedAUC.pdf}
	\caption{去偏成对损失的示意图} 
	\label{fig:auc}
\end{figure*}
%%%%%%%%%%%%%%%%%%%%%%%%%%%%%%%%%%%%%%%%%%%%%%%%%%%%%%%%%%%
有了式\eqref{eq:est}给出的优化目标“用户喜欢正例$\mathbf{x}_1$强于负例$\mathbf{x}_2$”的估计,可以很容易进行极大似然估计。按照BPR的做法,本章也最大化对数似然,那么去偏的成对损失(Debiased Pairwise Loss,DPL)的最终经验形式如下所示:
\begin{eqnarray}\label{eq:dpl}
	\mathcal{L}_\textsc{dpl}=- \frac{1}{|\mathcal{D}^+|\times |\mathcal{D}^u|} \sum_{\mathbf{x}^+ \in \mathcal{D}^+}\sum_{\mathbf{x} \in \mathcal{D}^u} && \ln \hat{P}_\textsc{pn}.
\end{eqnarray}

\section{算法实现与时间复杂度分析}
在使用BPR损失进行训练时,对于每个$(u,i)$对,需要一个负样本$j$,从而形成一个$(u,i,j)$的训练三元组。为了解决采样偏差,DPL损失要求对于每个$(u,i)$对,额外采样$M\geq 1$个正例和$N\geq1$个负例。遵循与BPR相同的数据条目格式,每个DPL数据条目被组织为$(u,i,i_1,i_2,\cdots,i_M,j_1,j_2,\cdots,j_N)$。这可以通过重写\verb|Dataloader|的\verb|collate_fn|函数,实现将一个批量大小为bs个交互$(u,i)$的mini-batch数据,改写为含有bs个数据条目为$(u,i,i_1,i_2,\cdots,i_M,j_1,j_2,\cdots,j_N)$的mini-batch数据。算法伪码如下:
\begin{algorithm}[!]
	\caption{Pytorch的collate\_fn函数重写}\label{4Alg2:1}
	\KwIn{含有bs个交互$(u,i)$的mini-batch数据$\mathcal{R}$,批量大小bs,物品集合$\mathcal{I}$,用户正例集合$\mathcal{I}_u^+$。}
	\KwOut{含有bs个条目为$(u,i,i_1,i_2,\cdots,i_M,j_1,j_2,\cdots,j_N)$的mini-batch数据$\mathcal{R}\prime$。}
	\For{$(u,i) \in \mathcal{R}$}
	{
		~~从正例集合$\mathcal{I}_u^+$中随机采样M个正例$i_1,i_2,\cdots,i_M$;\\
		从未标注样本集合$\mathcal{I}_u^+$中随机采样N个未标注样本$j_1,j_2,\cdots,j_N$;\\
		将交互$(u,i)$、M个正例、N个负例拼接为1个数据条目;\\
		将该数据条目添加到$\mathcal{R}\prime$中;
	}
	\KwResult{含有bs个条目为$(u,i,i_1,i_2,\cdots,i_M,j_1,j_2,\cdots,j_N)$的mini-batch数据$\mathcal{R}\prime$。}
\end{algorithm}

因此,每个mini-batch数据结构组织如下:
\begin{eqnarray}\label{eq:mini}
	\text{batch size}\left\{ \left[\begin{array}{cccccccccc}
		u^{1} & {i}   &i_{1} & i_{2} & \ldots &i_{M} &   j_{1}& j_{2} & \ldots &j_{N} \\
		u^{2} & i & i_{1}& i_{2} & \ldots &i_{M} &  j_{1}& j_{2} & \ldots &j_{N} \\
		\vdots & \vdots & \vdots & \vdots &\ddots & \vdots & \vdots & \vdots & \ddots & \vdots \\
		u^{bs} & i & i_{1}& i_{2} & \ldots &i_{M} &  j_{1}& j_{2} & \ldots &j_{N} 
	\end{array}\right]\right.
\end{eqnarray}
每个数据条目包括N个未标记的样本$j_1,j_2,\cdots,j_N$,可以计算相应的N个预测得分 $\hat{x}{j}^1,\hat{x}{j}^2,\cdots,\hat{x}_{j}^N$。对于给定的正例样本 $(u,i)$,其得分为 $\hat{x}_i$,可以基于N个负例样本的预测得分得到N个PU概率值。因此,基于mini-batch估计的PU概率值为
\[\hat{P}_\textsc{pu} = \frac{1}{N} \sum_{n=1}^{N}\sigma (\hat{x}_i - \hat{x}_{j}^n)\]
类似地,可以计算出M个正例$i_1,i_2,\cdots,i_M$ 对应的M个预测得分 $\hat{x}{i}^1,\hat{x}{i}^2,\cdots,\hat{x}_{i}^M$,相应地,计算PP概率值估计为:
\[\hat{P}_\textsc{pp} = \frac{1}{M} \sum_{m=1}^{M}\sigma (\hat{x}_i - \hat{x}_{i}^m)\]

因此,经过校正后的理想的优化目标为:
\[\hat{P}_\textsc{pn} = \frac{1}{\tau^-}
\hat{P}_\textsc{pu} - \frac{\tau^+}{\tau^-}\hat{P}_\textsc{pp} \]
和BPR一样最大化对数似然,只需要对上式求对数后,执行梯度下降,基于Pytorch矩阵运算的算法伪码如下:
\begin{algorithm}[!]
	\small
	\caption{去偏成对学习算法(DPL)伪代码}\label{4-Alg:2}
	\KwIn{Mini-batch数据 $\mathcal{R}\prime$如式\eqref{eq:mini}所示, 评分函数$s(\cdot)$,额外正例数量M,额外负例数量N,正例类先验$\tau^+$,批量大小bs。}
	\KwOut{模型参数$\Theta \in \mathbb{R}^d$}
	\For{迭代次数$t= 1,2,...,T$}{
		~~从训练集中抽样一个mini-batch数据 $\mathcal{R}\prime$如式\eqref{eq:mini}所示;\\
		正向计算评分scores = s ($\mathcal{R}\prime$);   \# [bs * (1+M+N)] \\
		pos\_scores = scores[: , : M+2];   \# [bs * (1+M)]  \\
		neg\_scores = scores[: , M+2:]; \# [bs * N] \\
		pu\_prob = sigmoid(pos\_scores[:, 0:1] - neg\_scores); \# [bs*N] \\ 
		pp\_prob = sigmoid(pos\_scores[:, 0:1] - pos\_scores[:,1:]);  \# [bs*M]  \\
		pn\_prob = pu\_prob.mean(dim=-1)/(1-tau) - $\tau^+ \cdot$ pp\_prob.mean(dim=-1)/(1-tau); \#[bs, ] \\
		dpl\_loss = - log (pn\_prob).mean(); \\
		根据dpl\_loss相对于$\Theta$的梯度更新参数;
	}
	\KwResult{用户和物品表示$\Theta$。}
\end{algorithm}

\textbf{复杂度}:首先分析基线方法BPR的复杂度,该复杂度与评分函数$s$有关。这里,以维度$d$的矩阵分解作为示例,这个分析可以很容易地扩展到其他模型。给定一个批量大小为$bs$的$(u,i,j)$的训练三元组,前向评分计算涉及到总共$2\times bs$个得分预测,因此时间复杂度为$\mathcal O(2 bs\times d)$。在反向传播中,最多更新$3 \times bs$个嵌入,涉及总共$5bs\times d$次操作,因此BPR的时间复杂度为$\mathcal O(bs\times d)$。类似地,对于DPL,一个小批量数据涉及到总共$(M+N+1)\times bs$个评分计算和$(M+N+2)\times bs$个嵌入更新,涉及总共$(2M+2N+3)\times bs\times d$个操作,时间复杂度为$\mathcal O(bs\times d)$,因为在实践中通常将$M$和$N$设置为小的常数,例如$M=3$和$N=3$。特别地,当$M=0$且$N=1$时,DPL涉及的操作数量与BPR相同。因此,相对于BPR,DPL具有严格的线性复杂度,没有任何mini-batch数据之外的计算或者存储开销。而前面的负采样算法,由于涉及计算用户每个训练三元组中用户$u$对应的评分向量,而推荐中,物品数量通常非常大,导致了时间复杂度的增加。

\section{去偏成对损失的理论分析}
DPL的主要思想是改进使用正例-未标记数据对(PU)计算的偏倚概率$P_\textsc{pu}$。通过采样额外的正例和负例样本进行来估计式\eqref{eq:est}所给出的理想优化目标“用户喜欢正例$\mathbf{x}_1$优于负例$\mathbf{x}_2$”的概率$\hat P_\textsc{pn}$,然后用其替代原始的偏倚概率估计$P_\textsc{pu}$。为了证明$\hat P_\textsc{pn}$是一个良好的估计量,首先证明$\hat P_\textsc{pn}$是AUC风险的无偏估计。
\begin{lemma}\label{lemma:auc} 设正例$\mathbf{x}^+$是从正例的类条件概率分布$p^+(\mathbf{x})$独立采样的样本, 未标注样本$\mathbf{x}$是从边缘分布$p(\mathbf{x}) = \tau^+ p^+(\mathbf{x})+ \tau^- p^-(\mathbf{x})$独立采样的样本。 那么公式~\eqref{eq:est}给出的$\hat P_\textsc{pn}$是无偏的AUC风险估计:
	\[\mathbb{E}\hat P_\textsc{pn} = R_{AUC}\]
	\begin{proof}
		\begin{eqnarray}
			\mathbb{E}\hat P_\textsc{pn} &=& \int_{\mathbf{x}^+} [\frac{1}{N\tau^-}
			\sum_{\mathbf{x} \in \mathcal{D}^u} \mathbb{E}_{\mathbf{x}\sim p}h(\mathbf{x}^+,\mathbf{x}) - \frac{\tau^+}{M\tau^-}\sum_{\mathbf{x}^\prime\in \mathcal{D}^+}\mathbb{E}_{\mathbf{x}^\prime \sim p^+}h(\mathbf{x}^+, \mathbf{x}^\prime)]p^+(\mathbf{x}^+)d\mathbf{x}^+ \\
			&=&\int_{\mathbf{x}^+} [\frac{1}{\tau^-}
			\mathbb{E}_{\mathbf{x}\sim p}h(\mathbf{x}^+,\mathbf{x}) - \frac{\tau^+}{\tau^-}\mathbb{E}_{\mathbf{x}^\prime \sim p^+}h(\mathbf{x}^+, \mathbf{x}^\prime)]p^+(\mathbf{x}^+)d\mathbf{x}^+ \label{eq:unbiasauc}.
		\end{eqnarray}
由于
		\begin{eqnarray}
			&&\frac{1}{\tau^-}\mathbb{E}_{\mathbf{x}\sim p}h(\mathbf{x}^+,\mathbf{x}) - \frac{\tau^+}{\tau^-}\mathbb{E}_{\mathbf{x}^\prime\sim p^+}h(\mathbf{x}^+, \mathbf{x}^\prime)\nonumber \\
			&=&\frac{1}{\tau^-} \int_\mathbf{x}h(\mathbf{x}^+,\mathbf{x})p(\mathbf{x})d\mathbf{x} - \frac{\tau^+}{\tau^-} \int_\mathbf{x^\prime}h(\mathbf{x}^+, \mathbf{x}^\prime)p^+(\mathbf{x}^\prime)d\mathbf{x^\prime} \nonumber \\
			&=&\frac{1}{\tau^-} \int_\mathbf{x}h(\mathbf{x}^+,\mathbf{x})[\tau^+p^+(\mathbf{x}) + \tau^-p^-(\mathbf{x}) ]d\mathbf{x} - \frac{\tau^+}{\tau^-} \int_\mathbf{x^\prime}h(\mathbf{x}^+, \mathbf{x}^\prime)p^+(\mathbf{x}^\prime)d\mathbf{x^\prime}  \label{eq:marginal} \\
			&=& \int_\mathbf{x}h(\mathbf{x}^+,\mathbf{x})p^-(\mathbf{x})d\mathbf{x} \label{eq:variable} \\
			&=& \int_\mathbf{x^-}h(\mathbf{x}^+,\mathbf{x}^-)p^-(\mathbf{x}^-)d\mathbf{x}^- \label{eq:pmin} 
		\end{eqnarray}
其中,公式~\eqref{eq:marginal}是通过边缘分布的全概率分解$p(\mathbf{x})= \tau^+p^+(\mathbf{x}) + \tau^-p^-(\mathbf{x})$得到, 公式~\eqref{eq:pmin}替换积分变量$\mathbf{x}$为$\mathbf{x}^-$以提高可读性。 将公式~\eqref{eq:pmin}带入公式~\eqref{eq:unbiasauc},有
		\begin{eqnarray}
			\mathbb{E}\hat{P}_\textsc{pn} &=& \int_{\mathbf{x}^+} \int_\mathbf{x}h(\mathbf{x}^+,\mathbf{x}^-)p^-(\mathbf{x}^-) p^+(\mathbf{x}^+)d\mathbf{x}^+d\mathbf{x}^- \label{eq:aucrisk}\\
			&=& R_{AUC}.
		\end{eqnarray}
证毕。
	\end{proof}	
\end{lemma}

如果将方程~\eqref{eq:aucrisk} 中的函数 $h$ 替换为 0-1 损失 $\mathbb{I}(\mathbf{x}^+,\mathbf{x}^-)$,那么方程~\eqref{eq:aucrisk}正好是AUC指标的严格定义。然而,由于 0-1 损失函数是离散的,因此在优化AUC指标时通常使用可微的替代损失函数。关于引理~\ref{lemma:auc} 的直观解释,请参考图~\ref{fig:auc}。需要说明的是,无偏估计的成立,不需要$M,N$趋近于无穷大作为必要条件,这得益于在成对学习的问题设置下,一个正例-未标注样本对(PU)所对应的标签情况可以被枚举(参考图\ref{fig:event})。

接下来,本章寻求一个理想的优化目标以供DPL近似。
\begin{definition}
对于固定的正例 $\mathbf{x}^+$,记$\mathbb{P}_\textsc{pn} = \mathbb{E}_{\mathbf{x}^-\sim p^-}h(\mathbf{x}^+,\mathbf{x}^-)$ 为用户喜欢正例$\mathbf{x}^+$强于负例$\mathbf{x}^-$的期望值。定义监督损失为这个对数似然对所有正例的期望值
	\begin{eqnarray}
		\mathcal{L}_\textsc{sup} =  -\mathbb{E}_{\mathbf{x}^+\sim p^+} \log \mathbb{P}_\textsc{pn}
	\end{eqnarray}
\end{definition}
最小化上述对数似然对所有正例样本的期望值,将导致最大化任意正例样本优于任意真负例样本的似然,这恰好是在完全监督数据下我们的优化目标。因此,$\mathcal{L}\textsc{sup}$是DPL近似的理想目标。引理~\ref{lemma:asy}证明了DPL估计量在M和N趋近无穷大时与该理想损失渐近一致。
\begin{lemma}\label{lemma:asy}
当$M,N \rightarrow +\infty$时,有
	\begin{eqnarray}
		\mathcal{L}_\textsc{dpl}  \rightarrow \mathcal{L}_\textsc{sup} 
	\end{eqnarray}
	\begin{proof}
勒贝格控制收敛定理\cite{tao:shi}表明,对于一个有界的可测函数序列$f_n$,有
		\begin{eqnarray}
			\lim\limits_{n\rightarrow \infty} \int_{\Omega} f_n =\int_{\Omega} \lim\limits_{n\rightarrow\infty}f_n \nonumber
		\end{eqnarray}
那么
		\begin{eqnarray}
			\lim\limits_{\substack{M,N \rightarrow +\infty}} \mathcal{L}_\textsc{dpl} 
			&=&-\lim\limits_{\substack{M,N \rightarrow +\infty}} \mathbb{E}_{\mathbf{x}^+\sim p^+}\log [\frac{1}{N\tau^-}
			\sum_{\mathbf{x} \in \mathcal{D}^u}h(\mathbf{x}^+,\mathbf{x})  \frac{\tau^+}{M\tau^-}\sum_{\mathbf{x}^\prime\in \mathcal{D}^+}h(\mathbf{x}^+, \mathbf{x}^\prime)]\nonumber \\
			&=&-\mathbb{E}_{\mathbf{x}^+\sim p^+}\lim\limits_{\substack{M,N \rightarrow +\infty}} \log [\frac{1}{N\tau^-}
			\sum_{\mathbf{x} \in \mathcal{D}^u}h(\mathbf{x}^+,\mathbf{x}) - \frac{\tau^+}{M\tau^-}\sum_{\mathbf{x}^\prime\in \mathcal{D}^+}h(\mathbf{x}^+, \mathbf{x}^\prime)]\nonumber \\
			&=&-\mathbb{E}_{\mathbf{x}^+\sim p^+}\lim\limits_{\substack{M,N \rightarrow +\infty}} \log [\frac{1}{N\tau^-}
			\sum_{\mathbf{x} \in \mathcal{D}^u}h(\mathbf{x}^+,\mathbf{x})  - \frac{\tau^+}{M\tau^-}\sum_{\mathbf{x}^\prime\in \mathcal{D}^+}h(\mathbf{x}^+, \mathbf{x}^\prime)]\nonumber \\
			&=&-\mathbb{E}_{\mathbf{x}^+\sim p^+}\log [\frac{1}{\tau^-}\mathbb{E}_{\mathbf{x}\sim p}h(\mathbf{x}^+,\mathbf{x}) \label{eq:lepn}- \frac{\tau^+}{\tau^-}\mathbb{E}_{\mathbf{x}^\prime\sim p^+}h(\mathbf{x}^+, \mathbf{x}^\prime)] \label{eq:epn1}
		\end{eqnarray}
应用公式~\eqref{eq:pmin}的结果,有
		\begin{eqnarray}
			&&\frac{1}{\tau^-}\mathbb{E}_{\mathbf{x}\sim p}h(\mathbf{x}^+,\mathbf{x}) - \frac{\tau^+}{\tau^-}\mathbb{E}_{\mathbf{x}^\prime\sim p^+}h(\mathbf{x}^+, \mathbf{x}^\prime)\nonumber \\
			&=& \int_\mathbf{x^-}h(\mathbf{x}^+,\mathbf{x}^-)p^-(\mathbf{x}^-)d\mathbf{x}^- \nonumber \\
			&=& \mathbb{E}_{\mathbf{x}^-\sim p^-}h(\mathbf{x}^+,\mathbf{x}^-) \nonumber\\
			&=&\mathbb{P}_\text{PN} \label{eq:epn}
		\end{eqnarray}
将公式~\eqref{eq:epn}带入公式\eqref{eq:epn1},有
		\[\mathcal{L}_\textsc{dpl}  \rightarrow \mathcal{L}_\textsc{sup},\]
证毕。
	\end{proof}
	
引理~\ref{lemma:asy}证明了当M和N趋近无穷大时,DPL损失$\mathcal{L}\textsc{dpl}$与理想的有监督损失$\mathcal{L}_\textsc{sup}$是渐近一致的。然而,在实际应用中,只可能使用有限大小的M和N,导致了经验估计$\hat{\mathcal{L}}_\textsc{dpl}$。接下来,引理~\ref{lemma:err}给出了经验估计$\hat{\mathcal{L}}\textsc{dpl}$的估计误差界$|\hat{\mathcal{L}}_\textsc{dpl}-\mathcal{L}_\textsc{sup}|$。
\end{lemma}
\begin{lemma}\label{lemma:err}
以至少$1-\delta$的概率,如下不等式成立
	\begin{eqnarray}
		&&|\hat{\mathcal{L}}_\textsc{dpl}-\mathcal{L}_\textsc{sup}| \leq \nonumber e^2\sqrt{\frac{2\pi}{N}} + e^2\tau^+\sqrt{\frac{2\pi}{M}}
	\end{eqnarray}
	\begin{proof}
不失一般性,证明使用余弦相似度作为得分函数来简化分析,这意味着所有嵌入被映射到半径为1的超球面上。$h$是一个将两个样本映射到一个sigmoid函数中的函数:$h(\mathbf{x}^+,\mathbf{x}) = \sigma (g(\mathbf{x}^+)- g(\mathbf{x}))$。由于$g(\cdot) \in [0,1]$,那么$-2 \leq g(\mathbf{x}^+)- g(\mathbf{x}) \leq 2$,且$ \frac{1}{1+e^{2}} \leq h(\mathbf{x}^+,\mathbf{x}) \leq \frac{1}{1+e^{-2}} $。	
		
对于固定正例$\mathbf{x}^+$, 记渐近优化目标和非渐近目标的被积函数之间的差值为$\triangle$:
		\begin{eqnarray}
			\triangle &=& | \log [ \frac{1}{N\tau^-}  \sum_{\mathbf{x}} h(\mathbf{x}^+,\mathbf{x})  -\frac{\tau^+}{M\tau^-} \sum_{\mathbf{x}^\prime} p(\mathbf{x}^+,\mathbf{x}^\prime)]  - \log\mathbb{E}_{\mathbf{x}^-\sim p^-}h(\mathbf{x}^+,\mathbf{x}^-)|\nonumber \\
			&=&| \log [ \frac{1}{N\tau^-}  \sum_{\mathbf{x}} h(\mathbf{x}^+,\mathbf{x})  -\frac{\tau^+}{M\tau^-} \sum_{\mathbf{x}^\prime} p(\mathbf{x}^+,\mathbf{x}^\prime)]  - \log\mathbb{E}_{\substack{\mathbf x \sim p(\mathbf x) \\ \mathbf x^\prime \sim p^+(\mathbf x)}} [ \frac{1}{\tau^-}h(\mathbf{x}^+,\mathbf{x}) - \frac{\tau^+}{\tau^-}h(\mathbf{x}^+,\mathbf{x}^\prime)]| \nonumber\\
			&=& | \log \frac{\frac{1}{N}  \sum_{\mathbf{x}} h(\mathbf{x}^+,\mathbf{x})  -\frac{\tau^+}{M} \sum_{\mathbf{x}^\prime} h(\mathbf{x}^+,\mathbf{x}^\prime)}{\mathbb{E}_{\substack{\mathbf x \sim p(\mathbf x) \\ \mathbf x^\prime \sim p^+(\mathbf x)}} [ h(\mathbf{x}^+,\mathbf{x}) - \tau^+h(\mathbf{x}^+,\mathbf{x}^\prime)]} | \nonumber
		\end{eqnarray}
由于$\mathbb{P}(|X|\geq \epsilon )=\mathbb{P}(X\geq \epsilon )+\mathbb{P}(-X\geq \epsilon )$,因此
		\begin{eqnarray}
			\mathbb{P}(\triangle \geq \epsilon) = \mathbf{I}(\epsilon) + \mathbf{II}(\epsilon) 
		\end{eqnarray}
其中
		\begin{eqnarray}
			\mathbf{I}(\epsilon) 
			&=& \mathbb{P} \left(\log \frac{\frac{1}{N}  \sum_{\mathbf{x}} h(\mathbf{x}^+,\mathbf{x})  -\frac{\tau^+}{M} \sum_{\mathbf{x}^\prime} h(\mathbf{x}^+,\mathbf{x}^\prime)}{\mathbb{E}_{\substack{\mathbf x \sim p(\mathbf x) \\ \mathbf x^\prime \sim p^+(\mathbf x)}} [ h(\mathbf{x}^+,\mathbf{x}) - \tau^+h(\mathbf{x}^+,\mathbf{x}^\prime)]} \geq \epsilon  \right) \\
			&\leq& \mathbb{P} \left( \frac{\frac{1}{N}  \sum_{\mathbf{x}} h(\mathbf{x}^+,\mathbf{x})  -\frac{\tau^+}{M} \sum_{\mathbf{x}^\prime} h(\mathbf{x}^+,\mathbf{x}^\prime)}{\mathbb{E}_{\substack{\mathbf x \sim p(\mathbf x) \\ \mathbf x^\prime \sim p^+(\mathbf x)}} [ h(\mathbf{x}^+,\mathbf{x}) - \tau^+h(\mathbf{x}^+,\mathbf{x}^\prime)]}-1 \geq \epsilon  \right) \label{eq:logx}\\
			&\leq& \mathbb{P} ( \frac{1}{N}  \sum_{\mathbf{x}} h(\mathbf{x}^+,\mathbf{x})  -\frac{\tau^+}{M} \sum_{\mathbf{x}^\prime} h(\mathbf{x}^+,\mathbf{x}^\prime)  \nonumber \\ &&-\mathbb{E}_{\substack{\mathbf x \sim p(\mathbf x) \\ \mathbf x^\prime \sim p^+(\mathbf x)}} [ h(\mathbf{x}^+,\mathbf{x}) - \tau^+h(\mathbf{x}^+,\mathbf{x}^\prime)] \geq \frac{\epsilon}{1+e^2}  ) \label{eq:emin}
		\end{eqnarray}
式~\eqref{eq:logx}是由于$\log x \leq x-1$对所有的$x>0$成立。式~\eqref{eq:emin}是由于 $\mathbb{E}_{\substack{\mathbf x \sim p(\mathbf x) \\ \mathbf x^\prime \sim p^+(\mathbf x)}} [ h(\mathbf{x}^+,\mathbf{x}) - \tau^+h(\mathbf{x}^+,\mathbf{x}^\prime)] =(1-\tau^+)\mathbb{E}_{\mathbf x \sim p^-(\mathbf x)}  h(\mathbf{x}^+,\mathbf{x}) \geq 1/(1+e^2)$。第二项类似
		\begin{eqnarray}
\mathbf{II}(\epsilon)&=& \mathbb{P} \left(\log 
			\frac{\mathbb{E}_{\substack{\mathbf x \sim p(\mathbf x) \\ \mathbf x^\prime \sim p^+(\mathbf x)}} [ h(\mathbf{x}^+,\mathbf{x}) - \tau^+h(\mathbf{x}^+,\mathbf{x}^\prime)]}{\frac{1}{N}  \sum_{\mathbf{x}} h(\mathbf{x}^+,\mathbf{x})  -\frac{\tau^+}{M} \sum_{\mathbf{x}^\prime} h(\mathbf{x}^+,\mathbf{x}^\prime)} \geq \epsilon  \right) \nonumber\\
			&\leq& \mathbb{P} \left( \frac{\mathbb{E}_{\substack{\mathbf x \sim p(\mathbf x) \\ \mathbf x^\prime \sim p^+(\mathbf x)}} [ h(\mathbf{x}^+,\mathbf{x}) - \tau^+h(\mathbf{x}^+,\mathbf{x}^\prime)]}{\frac{1}{N}  \sum_{\mathbf{x}} h(\mathbf{x}^+,\mathbf{x})  -\frac{\tau^+}{M} \sum_{\mathbf{x}^\prime} h(\mathbf{x}^+,\mathbf{x}^\prime)} -1 \geq \epsilon  \right) \label{eq:logxii}\nonumber\\
			&\leq& \mathbb{P} (\mathbb{E}_{\substack{\mathbf x \sim p(\mathbf x) \\ \mathbf x^\prime \sim p^+(\mathbf x)}} [ h(\mathbf{x}^+,\mathbf{x}) - \tau^+h(\mathbf{x}^+,\mathbf{x}^\prime)]  \nonumber \\ &&- [\frac{1}{N}  \sum_{\mathbf{x}} h(\mathbf{x}^+,\mathbf{x})  -\frac{\tau^+}{M} \sum_{\mathbf{x}^\prime} h(\mathbf{x}^+,\mathbf{x}^\prime)]  \geq \frac{\epsilon}{1+e^2}  ) \label{eq:eminii}
		\end{eqnarray}
综合式~\eqref{eq:emin}与式~\eqref{eq:eminii},有
		\begin{eqnarray}
\mathbb{P}(\triangle \geq \epsilon)
			&\leq& \mathbb{P} ( |\frac{1}{N}  \sum_{\mathbf{x}} h(\mathbf{x}^+,\mathbf{x})  -\frac{\tau^+}{M} \sum_{\mathbf{x}^\prime} h(\mathbf{x}^+,\mathbf{x}^\prime)  \label{eq:abs}\\ &&-\mathbb{E}_{\substack{\mathbf x \sim p(\mathbf x) \\ \mathbf x^\prime \sim p^+(\mathbf x)}} [ h(\mathbf{x}^+,\mathbf{x}) - \tau^+h(\mathbf{x}^+,\mathbf{x}^\prime)]| \geq \frac{\epsilon}{1+e^2}  ) \nonumber\\
			&=& \mathbb{P} (|[\frac{1}{N}  \sum_{\mathbf{x}} h(\mathbf{x}^+,\mathbf{x}) -\mathbb{E}_{\mathbf x \sim p(\mathbf x)}  h(\mathbf{x}^+,\mathbf{x}) ] \\
			&&-[\frac{\tau^+}{M} \sum_{\mathbf{x}^\prime} h(\mathbf{x}^+,\mathbf{x}^\prime)  -\mathbb{E}_{\mathbf x^\prime \sim p^+(\mathbf x)}  \tau^+h(\mathbf{x}^+,\mathbf{x}^\prime)]| \geq \frac{\epsilon}{1+e^2}  ) \nonumber\\
			&\leq& \mathbb{P} (|\frac{1}{N}  \sum_{\mathbf{x}} h(\mathbf{x}^+,\mathbf{x}) -\mathbb{E}_{\mathbf x \sim p(\mathbf x)}  h(\mathbf{x}^+,\mathbf{x})| \label{eq:abs1}\\
			&&+|\frac{\tau^+}{M} \sum_{\mathbf{x}^\prime} h(\mathbf{x}^+,\mathbf{x}^\prime)  -\mathbb{E}_{\mathbf x^\prime \sim p^+(\mathbf x)}  \tau^+h(\mathbf{x}^+,\mathbf{x}^\prime)| \geq \frac{\epsilon}{1+e^2}  ) \nonumber\\
			&\leq& \mathbf{III} (\epsilon) + \mathbf{IV} (\epsilon). \label{eq:abs2}
		\end{eqnarray}
其中
		\begin{eqnarray}
			\mathbf{III} (\epsilon) &=& \mathbb{P}(|\frac{1}{N}  \sum_{\mathbf{x}} h(\mathbf{x}^+,\mathbf{x}) -\mathbb{E}_{\mathbf x \sim p(\mathbf x)}  h(\mathbf{x}^+,\mathbf{x})| \geq \frac{\epsilon}{2(1+e^2)}  ) \\
			\mathbf{IV} (\epsilon) &=&\mathbb{P}(|\frac{\tau^+}{M} \sum_{\mathbf{x}^\prime} h(\mathbf{x}^+,\mathbf{x}^\prime)  -\mathbb{E}_{\mathbf x^\prime \sim p^+(\mathbf x)}  \tau^+h(\mathbf{x}^+,\mathbf{x}^\prime)|\geq \frac{\epsilon}{2(1+e^2)}  ) 
		\end{eqnarray}
式~\eqref{eq:abs1}是由于$|X-Y| \leq |X|+|Y|$成立,式~\eqref{eq:abs2}由于
		$\mathbb{P}(|X|+|Y| \leq \epsilon) \leq \mathbb{P}(|X| \leq \epsilon/2) + \mathbb{P}(|Y| \leq \epsilon/2)$。根据McDiarmid's不等式, 对于独立同分布的随机变量$X_{1},X_{2},\dots ,X_{n}$,如果存在界${\displaystyle c_{1},c_{2},\dots ,c_{n}}$和函数${\displaystyle f:{\mathcal {X}}_{1}\times {\mathcal {X}}_{2}\times \cdots \times {\mathcal {X}}_{n}\rightarrow \mathbb {R} }$,使得$|f(x_1,x_2,\cdots,x_i,\cdots,x_n)-f(x_1,x_2,\cdots,x_i^\prime,\cdots,x_n)|\leq c_i$对所有$i\in \{1,2,\cdots,n\}$成立,那么,对任意$\epsilon >0$,有如下不等式
		\begin{eqnarray}
			\mathbb{P}(|f(X_{1},X_{2},\ldots ,X_{n})-\mathbb {E} [f(X_{1},X_{2},\ldots ,X_{n})]|\geq \epsilon )   \leq 2\exp \left(-{\frac {2\epsilon ^{2}}{\sum _{i=1}^{n}c_{i}^{2}}}\right). \nonumber
		\end{eqnarray}
记未标记样本的得分 $g(\mathbf{x})$ 是一个随机变量,由于 $ \frac{1}{1+e^{2}} \leq h(\mathbf{x}^+,\mathbf{x}) \leq \frac{1}{1+e^{-2}} $,那么函数$f:{\mathcal {X}}_{1}\times {\mathcal {X}}_{2}\times \cdots \times {\mathcal {X}}_{n}\rightarrow \frac{1}{N}  \sum_{\mathbf{x}}  h(\mathbf{x}^+,\mathbf{x})$ 满足有界差异性质,边界为 ${\displaystyle c_{1}=c_{2}=\dots =c_{n}} = \frac{1}{N}\frac{e^2-1}{e^2+1}$,于是
		\begin{eqnarray}
			\mathbf{III}(\epsilon)  
			&\leq& 2\exp \left(-\frac{N\epsilon^2}{2(e^2-1)^2} \right)  \\
			&\leq& 2\exp \left(-\frac{N\epsilon^2}{2e^4} \right).\label{eq:p3}
		\end{eqnarray}
		记正样本的得分 $g(\mathbf{x}^\prime)$为随机变量,函数$f:{\mathcal {X}}_{1}\times {\mathcal {X}}_{2}\times \cdots \times {\mathcal {X}}_{m}\rightarrow \frac{\tau^+}{M} \sum_{\mathbf{x}}  h(\mathbf{x}^+,\mathbf{x}^\prime) $ 满足偏微分有界性质,边界为 ${\displaystyle c_{1}=c_{2}=\dots =c_{m}} = \frac{\tau^+}{M}\frac{e^2-1}{e^2+1}$, 于是
		\begin{eqnarray}
			\mathbf{IV}(\epsilon)  
			&\leq& 2\exp \left(-\frac{M\epsilon^2}{2(e^2-1)^2{\tau^+}^2} \right)  \\
			&\leq& 2\exp \left(-\frac{M\epsilon^2}{2e^4{\tau^+}^2}  \right).\label{eq:p4}
		\end{eqnarray}
将式~\eqref{eq:p3}和式~\eqref{eq:p4}带回式~\eqref{eq:abs2},有
		\begin{eqnarray}
			\mathbb{P}(\triangle \geq \epsilon|\mathbf{x}^+) \leq 2\exp \left(-\frac{N\epsilon^2}{2e^4} \right)+2\exp \left(-\frac{M\epsilon^2}{2e^4{\tau^+}^2}  \right). \label{eq:tail}
		\end{eqnarray}
为了获得$|\mathcal{L}_{\text{DPL}}(g) - \hat{\mathcal{L}}_\text{DPL}(g)| $的上界, 跟随 DCL~\cite{Chuang:2020:NIPS}的做法,通过应用Jensen's不等式,有
		\begin{eqnarray}
			|\hat{\mathcal{L}}_\textsc{dpl}-\mathcal{L}_\textsc{sup}|
			&=& \mathbb{E}_{\mathbf{x}^+} \log | \mathbf{I} - \mathbf{II}| \\
			&\leq& \mathbb{E}_{\mathbf{x}^+} \triangle\\
			&=&  \mathbb{E}_{\mathbf{x}^+} [\mathbb{E}_\epsilon[\triangle|\mathbf{x}^+]] \\
			&=&  \mathbb{E}_{\mathbf{x}^+} \left[\int_{0}^{+\infty} \mathbb{P}(\triangle \geq \epsilon|\mathbf{x}^+)d\epsilon \right] \label{eq:tail1} \\
			&\leq& \int_{0}^{+\infty}  2\exp \left(-\frac{N\epsilon^2}{2e^4} \right)+2\exp \left(-\frac{M\epsilon^2}{2e^4{\tau^+}^2}  \right) d \epsilon \nonumber\\
			&=& e^2\sqrt{\frac{2\pi}{N}} + e^2\tau^+\sqrt{\frac{2\pi}{M}}.
		\end{eqnarray}
在式~\eqref{eq:tail1} 中,外部的期望算子可以去掉,因为概率界对于所有固定的正例样本 $\mathbf{x}^+$ 都成立~\cite{Chuang:2020:NIPS}。证毕。
	\end{proof}
\end{lemma}

\section{实验评估}\label{pairsec:exp}
\subsection{实验设置}
\subsubsection{数据集}
本章在五个公开数据集上进行了实验:MovieLens-100k,MovieLens-1M,Yahoo!-R3,Yelp2018和Gowalla。前三个数据集包含用户评分,按照~\cite{Steffen:2009:UAI,Zhang:2013:SIGIR,Steffen:2014:WSDM} 的方法将所有有评分的物品转换为隐式反馈。对于每个数据集,随机将其中20\%的数据作为测试数据,剩下的80\%用于训练。表~\ref{4Table:Dataset} 给出了数据集统计的摘要信息。
\begin{table*}[h!]
	\centering
	\small
	\caption{数据集统计}\label{4Table:Dataset}
	\begin{tabular}{lrrrrrr}
		\toprule[1.2pt]
		~           & 用户数  & 物品数  &总交互数 & 训练集交互数  &测试集交互数&密度 \\ \cline{1-7}
		MovieLens-100k   &   943    &  1,682   &100,000&    80k	   & 20k &0.06304\\
		MovieLens-1M    &   6,040  &  3,952   &1,000,000&  800k     & 200k&0.04189  \\
		Yahoo!-R3       &   5,400  &  1,000  &182,000 &   146k      & 36k&0.03370\\
		Yelp2018       &   31,668  &  38,048&   1,561,406&   1,249k     & 312k&0.00130  \\
		Gowalla       &   29,858 &  40,981  &1,027,370 &   821k     & 205k&0.00084 \\
		\bottomrule[1.2pt]
	\end{tabular}
\end{table*}
\subsubsection{评估指标}
为了评估推荐的性能,我们采用常用的指标来评估Top-$K$推荐,这些指标包括精确度(P)、召回率(R)和归一化折损累积增益(NDCG),其中$K$的取值为5、10和20。由于这些指标的广泛使用,不在此处提供它们的定义。
\subsubsection{实验设置}
本章也使用两个推荐模型:经典的矩阵分解(MF)\cite{Koren:2009:Computer}和较新的轻量级图卷积网络(LightGCN)\cite{Xiangnan:2020:SIGIR}。前三个数据集的计算是在一台运行Windows 10操作系统的个人计算机上进行的,该计算机配备了2.1 GHz的CPU、一张RTX 1080Ti GPU和32 GB的RAM。后两个数据集的计算是在一台运行Linux操作系统的云服务器上进行的,该服务器配备了一颗Xeon(R) Platinum 8358P CPU、一张RTX A40 GPU和56GB的RAM。
\subsubsection{对比算法}
\begin{itemize}
	\item BPR~\cite{Steffen:2009:UAI}: 由于隐性反馈为positive-unlabeled数据,BPR从用户进行交互的$(u,i)$对中采样正例,而负例则从用户未进行交互的未标记数据$(u,j)$中抽取。
	\item InfoNCE~\cite{Oord:2018:arxiv}: 
	InfoNCE损失是机器学习中常用的损失函数,特别是在对比表示学习中。具体而言,InfoNCE衡量了正样本$\mathbf{x}^+$与一组负样本${\mathbf{x}_i^-}_{i=1}^N$之间的相似性,并应用了softmax函数:
	\begin{eqnarray}\label{eq:Info}
		\mathcal{L}_\text{InfoNCE} =  - \mathbb{E}_{\substack{\mathbf x^+ \sim p^+(\mathbf x) \\ \mathbf x_i^- \sim p^-(\mathbf x)}} \log\frac{\exp(g(\mathbf{x}^+))}{\exp(g(\mathbf{x}^+))+ \sum_{i=1}^{N}\exp( g(\mathbf{x}_i^-))} \nonumber
	\end{eqnarray}
	InfoNCE损失是噪声对比估计(NCE)以及BPR损失从一个负样本向N个负样本的推广。在实践中,由于负样本不可得,通常从未标记的样本中采样$\mathbf{x}_i^-$作为负样本。
	\item DCL~\cite{Chuang:2020:NIPS}: 由于未标记数据中存在虚假负例,DCL通过修正概率估计来进行虚假负例去偏。具体而言,DCL提出了一个估计量来修正$\mathcal{L}_\text{InfoNCE}$分母中的第二项 
	\begin{eqnarray}\label{eq:DCL}
		\mathcal{L}_\textsc{dcl} 
		&=&  - \mathbb{E}_{\substack{\mathbf x^+ \sim p^+(\mathbf x) \\ \mathbf x_i^- \sim p^-(\mathbf x)}}\log\frac{\exp(g(\mathbf{x}^+))}{\exp(g(\mathbf{x}^+))+ Ng} \nonumber
	\end{eqnarray}
	其中
	\begin{eqnarray}\nonumber
		g =  \frac{1}{N\tau^-}  (\sum_{i=1}^{N} \exp(g(\mathbf{x}_i) - N\tau^+ \cdot \frac{\sum_{j=1}^{K} \exp(g(\mathbf{x}^+_j)}{K} ) \label{Eq:DCLEstimator}
	\end{eqnarray}
	$\text{g}$估计量可以理解为真负样本分数的和。具体而言,$N\tau^+$伪负样本的数量的估计,而$\frac{\sum_{j=1}^{K} \exp(g(\mathbf{x}^+j)}{K}$估计了$K$个伪负样本分数的均值。因此,括号内的第二项对应于$N$个样本中所有伪负样本分数的和,将其从$N$个随机选择的未标记样本的分数和$\sum_{i=1}^{N} \exp(g(\mathbf{x}_i)$中减去,则对应于$N$个样本中所有真负样本的分数和。
	\item HCL~\cite{Robinson:2021:ICLR}: 使用了DCL的去偏框架,它还通过对每个随机选择的未标记样本进行加权来考虑困难样本挖掘,具体如下所示
	\begin{eqnarray}\label{eq:hcl}
		\omega_i^\textsc{Hcl} = \frac{g(\mathbf{x}^\prime_j)^\beta}{\frac{1}{N} \sum_{j=1}^{N}g(\mathbf{x}^\prime_j)^\beta}.
	\end{eqnarray}
	其中,$\beta$控制了挖掘难负例的程度。DCL是$\beta=0$时HCL的一个特例。
\end{itemize}
\subsection{实验结果}
\subsubsection{个性化推荐性能}
\begin{table*}[h!]
	\centering
	\caption{Top-k推荐性能比较}\label{4Table:Recommendation}
	\resizebox{1\textwidth}{!}{
		\begin{tabular}{lllccccccccccc}
			\toprule[1.2pt]
			\multirow{2}*{\textbf{数据集}} & \multirow{2}*{\textbf{模型}} & \multirow{2}*{\textbf{学习算法}} & \multicolumn{3}{c}{Top-5评估} &~& \multicolumn{3}{c}{Top-10评估}&~&\multicolumn{3}{c}{Top-20评估}\\ \cline{4-6} \cline{8-10} \cline{12-14}
			~ & ~ & ~ & Precision& Recall& NDCG& ~ &Precision& Recall& NDCG& ~ &Precision& Recall& NDCG \\ \hline
			
			\multirow{12}*{\textbf{MovieLens-100k}} & \multirow{6}*{\textbf{MF}} & BPR & 0.3900   &0.1301	&0.4143	&~&0.3363	&0.2164	&0.3967& ~&0.2724&0.3298&0.3962 \\
			~ & ~ & InfoNCE  &0.4081 & 0.1388 & 0.4324 & ~ & 0.3452 & 0.2266 & 0.4095 & ~ & 0.2793 & 0.3497 & 0.4118 \\
			~ & ~ & DCL &0.4168 & 0.1434 & 0.4458 & ~ & 0.3513 & 0.2291 & 0.4202 & ~ & 0.2835 & 0.3546 & 0.4207 \\ 
			~ & ~ & HCL  & 0.4263 & 0.1463 & 0.4539 & ~ & 0.3565 & 0.2323 & 0.426 & ~ & 0.2849 & 0.3564 & 0.4242 \\	
			~ & ~ &DPL(Proposed)    &0.4348 & 0.1523 & 0.4643 & ~ & 0.3635 & 0.2379 & 0.4356 & ~ & 0.2914 & 0.3588 & 0.4338 \\
			\cline{2-14}
			
			~ & \multirow{6}*{\textbf{LightGCN}}  & BPR & 0.3944 & 0.1231 & 0.4204 & ~ & 0.3346 & 0.2189 & 0.4017 & ~ & 0.2658 & 0.3281 & 0.3986 \\ 
			~ & ~ & Info\_NCE & 0.3924 & 0.1343 & 0.4209 & ~ & 0.3349 & 0.2183 & 0.4006 & ~ & 0.2679 & 0.3289 & 0.3976 \\ 
			~ & ~ & DCL & 0.3962 & 0.1367 & 0.4243 & ~ & 0.3361 & 0.2194 & 0.4022 & ~ & 0.2695 & 0.3329 & 0.4006 \\ 
			~ & ~ & HCL & 0.4197 & 0.1461 & 0.4501 & ~ & 0.3458 & 0.2256 & 0.4188 & ~ & 0.2802 & 0.3446 & 0.4182 \\
			~ & ~ & DPL(proposed) & 0.4333 & 0.1486 & 0.4627 & ~ & 0.3596 & 0.2344 & 0.4324 & ~ & 0.2919 & 0.3585 & 0.4331 \\ \hline\hline
			
			
			\multirow{12}*{\textbf{MovieLens-1M}} & \multirow{6}*{\textbf{MF}} & BPR & 0.3843    &0.0855	&0.4027	&~&0.3353	&0.1430	&0.3737& ~&0.2798&0.2244&0.3572 \\ 
			~ & ~ & InfoNCE & 0.3820 & 0.0879 & 0.4003 & ~ & 0.3339 & 0.1478 & 0.3728 & ~ & 0.2821 & 0.2358 & 0.3605 \\ 
			~ & ~ & DCL &  0.4009 & 0.0934 & 0.4209 & ~ & 0.3472 & 0.1546 & 0.3894 & ~ & 0.289 & 0.2423 & 0.3731\\ 
			~ & ~ & HCL & 0.4112 & 0.0969 & 0.4317 & ~ & 0.3552 & 0.1585 & 0.3991 & ~ & 0.2959 & 0.2475 & 0.3825 \\ 
			~ & ~ & DPL(proposed) & 0.4212 & 0.0998 & 0.4407 & ~ & 0.3624 & 0.1625 & 0.4071 & ~ & 0.2991 & 0.2518 & 0.3891 \\ 
			\cline{2-14}
			~ & \multirow{6}*{\textbf{LightGCN}} & BPR &0.4095&0.0953&0.4305&~&0.3512&0.1547&0.3985& ~&0.2915&0.2405&0.3781 \\ 
			~ & ~ & InfoNCE & 0.4121 & 0.0986 & 0.4386 & ~ & 0.359 & 0.1594 & 0.4041 & ~ & 0.2979 & 0.2482 & 0.3869 \\ 
			~ & ~ & DCL & 0.4104 & 0.0982 & 0.4291 & ~ & 0.3544 & 0.1597 & 0.3977 & ~ & 0.2965 & 0.2511 & 0.3842 \\ 
			~ & ~ & HCL & 0.4107 & 0.0948 & 0.4300 & ~ & 0.3514 & 0.1542 & 0.3950 & ~ & 0.2916 & 0.2413 & 0.3775 \\ 
			~ & ~ & DPL(proposed) & 0.4217 & 0.1003 & 0.4429 & ~ & 0.3620 & 0.1625 & 0.1866 & ~ & 0.2989 & 0.2511& 0.3896 \\\hline \hline
			
			\multirow{12}*{\textbf{Yahoo!-R3}} & \multirow{6}*{\textbf{MF}} & BPR & 0.1417 & 0.1052 & 0.1587 & ~ & 0.1064 & 0.1573 & 0.1641 & ~ & 0.0768 & 0.2259 & 0.1913 \\ 
			~ & ~ & Info\_NCE & 0.1429 & 0.1065 & 0.1615 & ~ & 0.1080 & 0.1601 & 0.1664 & ~ & 0.0786 & 0.2316 & 0.1952 \\ 
			~ & ~ & DCL &0.1454 & 0.1083 & 0.1635 & ~ & 0.1091 & 0.1618 & 0.1692 & ~ & 0.079 & 0.2327 & 0.1974  \\ 
			~ & ~ & HCL & 0.1460 & 0.1097 & 0.1638 & ~ & 0.1096 & 0.1628 & 0.1697 & ~ & 0.0792 & 0.2336 & 0.1976 \\ 
			~ & ~ & DPL(proposed) & 0.1491& 0.1091 & 0.1652 & ~ & 0.1108 & 0.1641 & 0.1712 & ~ & 0.0801 & 0.2351 & 0.2012 \\ 
			\cline{2-14}
			~ & \multirow{6}*{\textbf{LightGCN}} &BPR & 0.1479&0.1101&0.1693&~&0.1126&0.1669&0.1760&~& 0.0814&0.2389&0.2047 \\ 
			~ & ~ & Info\_NCE & 0.1417 & 0.1074 & 0.1676 & ~ & 0.1099 & 0.1633 & 0.1719 & ~ & 0.0798 & 0.2354 & 0.2007  \\ 
			~ & ~ & DCL & 0.1456 & 0.1092 & 0.1642 & ~ & 0.1089 & 0.1622 & 0.1697 & ~ & 0.079 & 0.2333 & 0.1982\\ 
			~ & ~ & HCL & 0.1412 & 0.1139 & 0.1718 & ~ & 0.113 & 0.1683 & 0.1776 & ~ & 0.0812 & 0.2394 & 0.2059 \\ 
			~ & ~ & DPL(proposed) & 0.1504 & 0.1111 &  0.1697 & ~ & 0.1131 & 0.1670  & 0.1757 & ~ & 0.0825 & 0.2412 & 0.2054\\ \hline\hline
			
			
			\multirow{12}*{\textbf{Yelp2018}} & \multirow{6}*{\textbf{MF}} & BPR & 0.0398 & 0.0228 & 0.0435 & ~ & 0.0339 & 0.0389 & 0.0456 & ~ & 0.0284 & 0.065 & 0.0538 \\ 
			~ & ~ & Info\_NCE  & 0.0429 & 0.0246 & 0.047 & ~ & 0.0365 & 0.0417 & 0.0491 & ~ & 0.0305 & 0.07 & 0.058 \\ 
			~ & ~ & DCL & 0.0486 & 0.0278 & 0.0531 & ~ & 0.0410 & 0.0466 & 0.0552 & ~ & 0.0342 & 0.0777 & 0.0648 \\
			~ & ~ & HCL & 0.0515 & 0.0305 & 0.0566 & ~ & 0.0459 & 0.0541 & 0.0622 & ~ & 0.0383 & 0.0894& 0.0736 \\ 
			~ & ~ & DPL(proposed) & 0.0543 & 0.0325 & 0.0595 & ~ & 0.0463 & 0.0551 & 0.0630 & ~ & 0.0389 & 0.0914 & 0.0749 \\
			\cline{2-14}
			~ & \multirow{6}*{\textbf{LightGCN}} & BPR & 0.0556 & 0.0330 & 0.0610 & ~ & 0.0473 & 0.0560 & 0.0644 & ~ & 0.0391 & 0.0914 & 0.0757 \\ 
			~ & ~ & Info\_NCE & 0.0553 & 0.0329 & 0.0607 & ~ & 0.0473 & 0.0558 & 0.0642 & ~ & 0.0390 & 0.0911 & 0.0754 \\ 
			~ & ~ & DCL & 0.0559 & 0.0331 & 0.0612 & ~ & 0.0472 & 0.0557 & 0.0642 & ~ & 	0.0391 & 0.0914 & 0.0756 \\ 
			~ & ~ & HCL & 0.0563 & 0.0335 & 0.0617 & ~ & 0.0477 & 0.0564 & 0.0648 & ~ & 0.0393 & 0.0920 & 0.0760 \\  
			~ & ~ & DPL(proposed) & 0.0604 & 0.0364 & 0.0657 & ~ & 0.0513 & 0.0615 & 0.0696 & ~ & 0.0423 & 0.1003 & 0.0821 \\\hline\hline
			
			
			\multirow{12}*{\textbf{Gowalla}} & \multirow{6}*{\textbf{MF}} & BPR & 0.0728 & 0.0748 & 0.1000 & ~ & 0.0555 & 0.1116 & 0.1063 & ~ & 0.0414 & 0.1625 & 0.1209 \\ 
			~ & ~ & Info\_NCE & 0.0739 &0.0757 & 0.1016 & ~ & 0.0560 & 0.1122 & 0.1076 & ~ & 0.0422 & 0.1650 & 0.1230\\ 
			~ & ~ & DCL & 0.0746 & 0.0769 &0.1023 & ~ & 0.0568 & 0.1147 & 0.1088 & ~ & 0.0426 & 0.1664 & 0.1238 \\ 
			~ & ~ & HCL & 0.0755 & 0.0774 & 0.1035 & ~ & 0.0574 & 0.1151 & 0.1098 & ~ & 0.0432& 0.1693 & 0.1256 \\ 
			~ & ~ & DPL(proposed) & 0.0815 & 0.0827 & 0.1100 & ~ & 0.0628 & 0.1243& 0.1174 & ~ & 0.0473 & 0.1815 & 0.1340 \\
			\cline{2-14}
			~ & \multirow{6}*{\textbf{LightGCN}} & BPR & 0.0735 & 0.0753 & 0.1007 & ~ & 0.0560 & 0.1119 & 0.1069 & ~ & 0.0419 & 0.1641 & 0.1218 \\ 
			~ & ~ & Info\_NCE & 0.0743 & 0.0760 & 0.1022 & ~ & 0.0566 & 0.1132& 0.1084 & ~ & 0.0423 & 0.1649 & 0.1231\\ 
			~ & ~ & DCL & 0.0748 & 0.0763 & 0.1027 & ~ & 0.0569 & 0.1132 & 0.1088 & ~ & 0.0424 & 0.1656 & 0.1236 \\ 
			~ & ~ & HCL & 0.0794 & 0.0804 & 0.1084 & ~ & 0.0608 &0.1199 & 0.1147 & ~ & 0.0453 & 0.1740 & 0.1319 \\ 
			~ & ~ & DPL(proposed) & 0.0867 &  0.0891 & 0.1164 & ~ &  0.0662 &  0.1329 & 0.1242 & ~ & 0.0494 & 0.1936 & 0.1417  \\ \hline
			\bottomrule[1.5pt]
			
		\end{tabular}
	}
\end{table*}
从表~\ref{4Table:Recommendation}中可以得出第一个观察结果,即DPL取得了最佳性能。与没有去偏机制的BPR和InfoNCE相比,DPL显示出了显著的改进,表明在隐性反馈数据中纠正偏倚的概率估计的必要性。与具有去偏机制的DCL和HCL相比,DPL也取得了显著的改进,主要是由于DPL在成对学习问题设置中的去偏机制的优势。在成对学习问题设置中,未标记样本的数量为1,其对应的真实标签可以枚举(参见图~\ref{fig:event}),进而可以获得“用户更喜欢正例强于负例的概率”的无偏估计。然而,在N个未标记样本的情况下,根据二项式定理,真实标签有$2^N$种可能的结果,使得枚举每种情况变得困难。因此,DCL和HCL的去偏机制主要依赖于数值逼近。

第二个观察结果是,作为InfoNCE的一个特例,BPR损失在性能上稍逊于具有N个负样本的InfoNCE,特别是在大型稀疏数据集上。这是因为较大的负样本数量N会更紧密地推高互信息下界~\cite{Oord:2018:arxiv}。然而,在PU数据集中,较大的N并不总是更好,因为较大的N通常会导致对难样本(即在嵌入空间中与锚点(即用户嵌入)更接近的样本)赋予更大的梯度值。如果这个样本是一个伪负例,较大的梯度值会严重损害模型的性能。

第三个观察结果是,具有去偏机制的DCL、HCL和DPL通常优于缺乏去偏机制的BPR和InfoNCE,尤其是在具有较高正类先验的数据集上。此外,HCL通过将高分数的难样本分配更高的权重,给负样本分配较大的梯度值,在去偏的基础上通过隐式难负样本挖掘取得良好的结果。然而,应该谨慎调整相应的困难样本挖掘参数$\beta$,以防止伪负例对模型性能的伤害。这是因为如果困难样本是一个伪负例,模型性能将受到损害,但如果是一个真负例,模型性能将从难样本中受益。这种现象在协同过滤中被称为“探索与利用的权衡”\cite{Bin:2023:ICDE},在计算机视觉中被称为“均匀性与对齐性的困境”\cite{Feng:2021:CVPR}。
\subsubsection{超参数分析}
\begin{figure*}[h!]
	\centering
	\includegraphics[width=\textwidth]{4-parameter.png}
	\caption{不同超参数的影响} %需要说明的是无偏估计,不需要m,n趋近于无穷大,这得益于在成对学习的问题设置下,一个pu构成的datapair所对应的标签情况可以被枚举。其对应的校正项目有两个正例构成,我们能refer to 图1的阐释性例子说明两个正例构成的补偿项目的由来。
	\label{Fig:parameter}
\end{figure*}
M参数的影响:参数M控制用于纠正采样偏差的额外正例数量。当M=0时,没有去偏机制。对于MF模型,较大的M和N通常会导致性能提升,从M=0到M=1观察到显著的性能提升,突出了使用额外正例进行去偏的重要性。然而,当M和N超过一个小的常数时,性能不再提高。这是因为进一步增加M和N所带来的估计准确性的边际收益变得有限,正如之前的理论分析所述。

N参数的影响:参数N控制用于计算PU概率的负例数量。与MF模型类似,增加M始终会提高LightGCN的性能。然而,当负例数量N增加时,模型性能意外地下降。这个非预期的结果可能应该归因于较大的N会降低难负例样本的梯度值。根据詹森不等式,$-\log(\frac{1}{N} \sum_{i=1}^N P_n) \leq -\frac{1}{N} \sum_{i=1}^N \log P_n $;因此,过高的N值削弱了难负样本对学习算法的贡献。理论分析假设得分$g(\cdot)$是独立同分布(i.i.d.)变量,并且估计误差随着N的增加而减小。然而,基于图神经网络编码器的聚合机制严重影响了$g(\cdot)$值的独立同分布性质。因此,当推荐模型为难以优化的神经网络模型如LightGCN时,应当设置较小的N值。

正类先验$\tau^+$的影响:随着$\tau^+$增加,MF和LightGCN模型的性能呈现出倒U型曲线,先增加后减少。这是因为设置过低或过高的$\tau^+$值可能导致公式\eqref{eq:est}的估计结果偏倚。一种常见的设置$\tau^+$值的方法是将观察到的正交互作用的数量$|\mathcal{D}^+|$视为伯努利试验的结果,该试验总共发生$|\mathcal{U}|\times |\mathcal{I}|$次,成功的次数为观察到的交互作用数量。数据集的密度$|\mathcal{D}^+|/(|\mathcal{U}|\times |\mathcal{I}|)$可以作为设置$\tau^+$值的参考。然而,需要注意的是,基于数据集密度设置的$\tau^+$值也是一种偏倚估计,低估了真实值,因为数据中的所有未观察到的(u, i)对都被视为负样本,导致成功次数的低估。详细讨论可参考~\cite{Jain:2016:NIPS,Christoffel:2016:ACML}。

\subsubsection{估计校正 vs 负采样}
负采样和估计校正代表了解决采样偏差问题的两种不同技术方法。负采样的核心思想是在模型外选择理想的样本并将其喂入模型训练,而估计校正的思想是使用随机样本,但对损失函数进行修正。从贝叶斯统计的角度来看,负采样利用了两种类型的信息:先验信息,如物品类别和流行度,这是静态的信息,用于采样用户不喜欢的负样本;样本信息,如分数和排名位置,这是动态的信息,在模型训练过程中不断调整,并用于采样嵌入接近锚点嵌入(得分较高)的难样本。各种负采样算法之间的差异在于它们如何利用和处理这两种类型的信息。上一章的贝叶斯负采样算法制定了后验概率意义上的负信号测度,并提出了理论上最优的采样规则。表~\ref{Table:vsns}中比较了DPL和BNS的性能。
\begin{table*}[h!]
	\centering
	\caption{去偏成对学习算法与负采样算法的性能比较}\label{Table:vsns}
	\resizebox{1\textwidth}{!}{
		\begin{tabular}{lllccccccccccc}
			\toprule[1.2pt]
			\multirow{2}*{\textbf{数据集}} & \multirow{2}*{\textbf{模型}} & \multirow{2}*{\textbf{方法}} & \multicolumn{3}{c}{Top-5} &~& \multicolumn{3}{c}{Top-10}&~&\multicolumn{3}{c}{Top-20}\\ \cline{4-6} \cline{8-10} \cline{12-14}
			~ & ~ & ~ & Precision& Recall& NDCG& ~ &Precision& Recall& NDCG& ~ &Precision& Recall& NDCG \\ \hline
			
			\multirow{4}*{\textbf{MovieLens-100k}} & \multirow{4}*{\textbf{MF}} & BPR & 0.3900   &0.1301	&0.4143	&~&0.3363	&0.2164	&0.3967& ~&0.2724&0.3298&0.3962 \\
			~ & ~ &BNS   &0.4205 &0.1467	&0.4558&~	&0.3463	&0.2290	&0.4217& ~&0.2762&0.3466& 0.4176\\	
			~ & ~ &DPL(Proposed)    &0.4348 & 0.1523 & 0.4643 & ~ & 0.3635 & 0.2379 & 0.4356 & ~ & 0.2914 & 0.3588 & 0.4338 \\
			~ & ~ &DPL with Hard Samples   &0.4401 & 0.1579 & 0.4692 & ~ & 0.3713 & 0.2407 & 0.4395 & ~ & 0.2940 & 0.3592 & 0.4351 \\\hline \hline 
			
			
			\multirow{4}*{\textbf{MovieLens-1M}} & \multirow{4}*{\textbf{MF}} & BPR &0.3843    &0.0855	&0.4027	&~&0.3353	&0.1430	&0.3737& ~&0.2798&0.2244&0.3572\\ 
			~ & ~ & BNS&0.4207	&0.1062	&0.4324&~	&0.3518	&0.1703	&0.4191& ~&0.3045&0.2614&0.4002 \\
			~ & ~ & DPL(proposed) & 0.4212 & 0.0998 & 0.4407 & ~ & 0.3624 & 0.1625 & 0.4071 & ~ & 0.2991 & 0.2518 & 0.3891  \\
			~ & ~ & DPL with Hard Samples & 0.4251 & 0.1012 & 0.4412 & ~ & 0.3649 & 0.1701 & 0.4151 & ~ & 0.3012 & 0.2539 & 0.3922  \\\hline 
			\bottomrule[1.5pt]
		\end{tabular}
	}
\end{table*}
\begin{figure}[h!]
	\centering
	\includegraphics[width=0.6\textwidth]{4-runtime.png}
	\caption{去偏成对学习算法和负采样算法的运行时间比较} %需要说明的是无偏估计,不需要m,n趋近于无穷大,这得益于在成对学习的问题设置下,一个pu构成的datapair所对应的标签情况可以被枚举。其对应的校正项目有两个正例构成,我们能refer to 图1的阐释性例子说明两个正例构成的补偿项目的由来。
	\label{Fig:runtime}
\end{figure}

\textit{性能比较}:
DPL在MovieLens100k数据集上取得了更好的性能,并且在MovieLens1M数据集上与BNS表现相似,这表明基于校正估计的DPL作为一种替代负采样算法是可行的。此外,我们发现使用困难、得分较高的未标记样本来训练DPL可以带来一些性能改进。当使用困难样本训练DPL时,需要设置相对较大的$\tau^+$值。在表~\ref{Table:vsns}中,分别将$\tau^+$设置为0.3和0.25,这比数据集本身的密度要高得多。这是因为使用困难样本训练DPL会增加模型遇到误分类负样本的概率,这相当于人为地改变了喂入模型的训练样本的正类先验。

\textit{运行时间}: 运行时间是在一台配备2.1 GHz CPU、RTX 1080Ti GPU和32 GB RAM的个人计算机上进行测试的。图~\ref{Fig:runtime}显示了三种算法在五个数据集上进行一次epoch训练的运行时间,其中协同过滤模型固定为MF,批大小固定为1024。在计算经验分布函数时,BNS只采用mini-batch内的样本近似计算。如图~\ref{Fig:runtime}所示,DPL的实际运行时间仅略长于BPR,这与之前的时间复杂性分析一致。即使采用批内样本近似以节省计算成本,BNS的运行时间仍然是BPR和DPL的3-5倍。这是因为动态负采样需要预测得分来指导负采样,但基于GPU的批计算在进行前向传播以预测得分之前需要固定的负样本。因此,动态负采样通常通过将额外的负样本作为候选项加载到小批量数据中来实现,从而导致额外的计算成本。此外,一些最先进的动态负采样算法在前几个训练epoch中需要样本的排名位置或预测得分的方差~\cite{Ding:2019:NeurIPS},这要求模型在进行前向传播时计算整个用户-物品评分矩阵的预测得分,而不仅仅是小批量数据内的得分,从而导致指数级的时间复杂度。

总之,在具有丰富的辅助信息的场景中,建议使用可以灵活组合先验信息和模型信息的负采样算法。在没有可用的辅助信息进行监督的场景中,建议使用DPL方法进行损失修正。
\section{本章小结}
本章专注于解决来自正样本-未标记隐性反馈数据的采样偏差问题,但采用了与显式的负采样不同的技术方法。具体而言,本章提出了一种用于隐性反馈采样偏差的估计校正方法,从而为成对学习提供了一种修正的损失函数,称为去偏对数损失(DPL)。DPL的关键思想是纠正由伪负例导致的概率估计偏差,从而修正梯度以近似完全监督数据的梯度。所提出的目标函数易于实现,不需要额外的辅助信息进行监督,也不需要过多的存储和计算开销。在保持严格的相对于BPR的线性时间复杂度情况下,DPL取得了相对于负采样相当或更优的表现。

式\eqref{eq:infonce1}展示了成对损失和对比损失的联系,成对损失函数是负例个数N=1的对比损失的特例。本章聚焦于成对损失的去偏研究,DPL的去偏机制的核心可以参考图~\ref{fig:event},在只有一个负例的情况下,未标注样本的真实标签有且只有两种可能性,很容易被枚举出来。在自监督对比学习中,一个广泛的共识是负例个数N越大,学到的对比表示在下游任务中泛化表现越好\cite{Oord:2018:arxiv,Chuang:2020:NIPS,Robinson:2021:ICLR},这是由于更大的N推高了已知数据和待预测数据互信息的下界。但是,随着未标注样本N的增长,真实的标签的可能性指数级增长为$2^N$种,难以被枚举出来,导致本章的去偏思路则难以向更大的负例数推广。那么如何将去偏差的方法向更一般的对比损失推广?此外,对于在自监督对比学习任务而言,除了去偏以防止伪负例被推远,还有一个重要的硬负例挖掘任务,使真负例被推远。本章只考虑到前者。如何同时兼顾伪负例去偏和硬负例挖掘任务?下一个章节将针对这两个问题给出解决方案。

