%%% mode: latex
%%% TeX-master: t
%%% End:

\chapter{总结与展望}
\label{cha:conclusion}

\section{全文总结}
\label{sec:conclusion}
搜索推荐场景面临较为突出的正例弱监督、负例无监督的问题,使得从推荐数据集中学习良好的样本特征面临严重的挑战,并对排序预测任务产生了不利影响。对比学习利用了数据本身的结构和特性,无需额外的标注信息,就可以从数据中学习到有用的特征表示,是自监督学习的一种重要的实现方式,也是提升推荐算法准确性的有效途径。本文从推荐场景中正例弱监督、负例无监督的问题出发,针对自监督对比学习的三个关键组件正例、负例、损失函数做出了改进,提出了相应的正例去噪算法、负例采样算法和损失函数校正算法,并且将算法从只有一个负例的成对学习向更具一般性的多个负例的对比学习推广。本文的创新和贡献如下:

针对正例弱监督引发的伪正样本问题,提出了自适应学习的成对排序算法,用于从含噪成对比较的数据中学习个性化排序。分析了隐式反馈弱监督的本质是数据不完全(incomplete data),并把从含噪成对比较的数据中学习排序的问题形式化为一个含有隐变量的最大后验估计问题。引入一个隐变量作为衡量交互置信度的新指标,这个隐变量与用户物品的特征表示一起作为模型参数一起端到端地学习。所提出的算法采用期望最大化框架:在期望步骤中使用贝叶斯推断来估计交互的置信度指标;在最大化步骤中固定置信度指标,更新参数学习用户和物品的特征表示。该算法在合成的噪声数据集上表现出了更好的鲁棒性,在真实的数据集上表现出更高的推荐准确性。

针对负例无监督引发的伪负样本问题,提出了贝叶斯最优负采样算法,用于从未标记的数据中采样高质量的负例,以提升对比学习的训练效果并提升推荐精度。提出了贝叶斯最优负采样准则,这是最小化经验采样风险的理论最优的采样规则。该方法指定了后验概率意义上的负信号测度,结合了静态的先验信息和动态的样本信息,统一了现有的两种提取负信号的两种主要范式,并为使用辅助信息建模先验概率的方法提供了灵活的接口。该算法在采样误差率和采样样本信息量以及推荐准确性上优于同类方法。

针对负例无监督导致的有偏成对损失函数,提出了去偏成对学习算法,以近似完全监督数据的成对损失函数,以获得接近于有监督学习的泛化性能。该方法通过校正伪负样本导致的概率估计偏差,从而修正梯度以近似完全监督数据的梯度。所提出的目标函数易于实现,不需要额外的辅助信息进行监督,也不需要过多的存储和计算开销。在保持严格的相对于成对学习的线性时间复杂度情况下,去偏成对学习算法取得了更高的推荐准确性。

针对负例无监督导致的有偏对比损失函数,提出了贝叶斯自监督对比学习算法,通过重要性权重来纠正从无标签数据中随机选择的负样本引入的偏差。从单个权重来看,该方法提供了样本是真负例的后验概率估计,可以灵活统一地执行硬负例挖掘和伪负例去偏任务这两个冲突任务。从损失函数值来看,该方法提供了有监督数据下对比损失的一致估计,从而提升下游任务的泛化性能。在数值实验、个性化推荐和图像分类等多个任务上验证了贝叶斯自监督对比学习算法的有效性。只需要简单地修改损失函数,而无需额外的存储和计算开销。

\section{后续工作展望}
\label{sec:conclusion}
针对推荐数据集弱监督问题,本文围绕正例、负例和损失函数这三个对比学习的关键组件展开研究,并提出了正例去噪算法、负例采样算法和损失函数校正算法。然而,从弱监督的隐式反馈数据集中学习良好的样本特征仍存在一些值得关注的问题,从参数更新方式、图数据增强方式和网络架构设计三个方面总结如下:

在参数更新方式方面,以MOCO、BYOL为代表的模型采用异步更新策略,即分别采用梯度法和动量法更新两个网络分支参数。然而,为何动量法的更新策略有效,仍是一个自监督对比学习尚需回答的问题。本文的损失函数校正虽然针对同步更新策略设计,但实质上通过校正损失函数来调节梯度,改变参数更新方式。深刻理解动量法工作机理,将基于梯度法的参数更新改进方式推广至动量法,设计更优的动量更新策略以提升自监督对比学习效果,是自监督对比学习领域一个值得探索的重要理论和实践问题。

在数据增强方面,获取可靠的图数据增强是解决正例噪声的有效手段,比本文第三章提出的含有隐变量的最大后验估计更高效。由于图数据的非欧几里得特性,难以像图像数据那样通过随机剪裁、高斯模糊等操作获得语义不变的数据增强。在数据增强方法方面,目前主流的基于随机过程的图数据增强可能会损失图中重要的结构信息,如何获取稳定、可靠且保留重要结构信息的增强图,是一个重要的问题。从对比的方式来看,推荐中主流的方式是将原图和增强图进行对比,即只进行一次数据增强。如何组织互补的图数据增强方式,使得增强图之间进行相互对比,从而受益于更密集的数据增强,是提升推荐性能的重要途径。

在网络架构设计方面,采用异步异构的网络架构设计无需负样本,有效地避免了伪负例的错误梯度和网络坍塌,取得了令人惊艳的结果。在推荐中,样本特征提取所依赖的图卷积网络不同于提取图像数据的残差网络,具有结构上的特殊性。如何设计符合推荐场景的异步更新策略和网络架构,是一个值得探索的问题。


