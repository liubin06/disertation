%%% mode: latex
%%% TeX-master: t
%%% End:

\chapter{总结与展望}
\label{cha:conclusion}

\section{全文总结}
\label{sec:conclusion}
搜索推荐场景面临较为突出的数据标注不完全的问题,使得从弱监督的隐式反馈数据集中学习良好的样本特征面临严重的挑战。自监督推荐(SSR)对用户行为数据进行某种变换和处理提取自监督信号,有助于缓解数据稀疏、泛化误差、虚假相关和对抗攻击问题,受到了广泛的关注。基于对比学习的推荐算法是自监督推荐的主要实现方式。本文从自监督推荐场景出发,针对对比型推荐算法的三个关键组件正例、负例、损失函数做出了改进,提出了相应的正例去噪算法、负例采样算法和对比损失校正算法。本文的创新和贡献如下:

针对伪正样本问题,提出了自适应正例去噪的推荐算法,以降低伪正例的负面影响,从而学习更准确的特征表示。分析了隐式反馈的本质是数据不完全(incomplete data),并把从含噪成对比较的数据中学习排序的问题形式化为一个含有隐变量的最大后验估计问题。引入一个隐变量作为衡量交互置信度的新指标,这个隐变量与用户物品的特征表示一起作为模型参数一起端到端地学习。所提出的算法采用期望最大化框架:在期望步骤中使用贝叶斯推断来估计交互的置信度指标;在最大化步骤中固定置信度指标,更新参数学习用户和物品的特征表示。该算法在合成的噪声数据集上表现出了更好的鲁棒性,在真实的数据集上表现出更高的推荐准确性。

针对伪负样本问题,提出基于最优负采样准则的推荐算法,用于从未标记的数据中采样高质量的负例,以提升对比型推荐算法的训练效果并提升推荐精度。该方法指定了后验概率意义上的负信号测度,结合了静态的先验信息和动态的样本信息,较好地统一了现有的两种提取负信号的两类方法,并为使用辅助信息建模先验概率的方法提供了灵活的接口。进一步地提出了最优负采样准则,这是最小化经验采样风险的理论最优的采样规则,核心思想是选择一个未标注样本最大化两个训练轮次之间排序列表的AUC增益。该算法在采样误差率和采样样本信息量以及推荐准确性上优于同类方法。

针对成对学习优化目标偏离问题,提出了基于去偏成对损失的推荐算法。去偏成对损失通过近似完全监督设置下的成对损失函数,以获得接近于有监督学习的泛化性能。该方法通过校正伪负样本导致的概率估计偏差,从而修正梯度以近似完全监督数据的梯度。所提出的目标函数易于实现,不需要额外的辅助信息进行监督,也不需要过多的存储和计算开销。在保持严格的相对于成对学习的线性时间复杂度情况下,去偏成对学习算法取得了更高的推荐准确性。

针对对比学习优化目标偏离问题,提出了基于贝叶斯自监督对比损失的推荐算法。贝叶斯自监督对比损失通过重要性权重来纠正从无标签数据中随机选择的负样本引入的偏差。从单个权重来看,该方法提供了样本是真负例的后验概率估计,可以灵活统一地执行硬负例挖掘和伪负例去偏任务这两个冲突任务。从损失函数值来看,该方法提供了有监督数据下对比损失的一致估计,从而提升下游任务的泛化性能。在数值实验、个性化推荐和图像分类等多个任务上验证了贝叶斯自监督对比损失函数的有效性。

\section{后续工作展望}
\label{sec:conclusion}
本文围绕对比学习的关键组件,针对自监督设置下基于对比学习的推荐算法的正例、负例和损失函数做出了改进,并提出了相应的正例去噪算法、负例采样算法和损失函数校正算法。在现有研究的基础上,从自监督信号的获取、网络参数更新方式、编码器设计和网络架构设计四个方面展望后续的工作。

(1)\textit{在自监督信号提取方面},获取更可靠的自监督信号是提升推荐性能的主要途径,也是是解决噪声的有效手段。对于图对比学习广泛使用的“图数据增强的语义不变性”的自监督信号而言,由于图数据的非欧几里得特性,难以像图像数据那样通过随机剪裁、高斯模糊等操作获得语义不变的数据增强,如何获取稳定、可靠且保留重要结构信息的增强图,是一个重要的问题。此外,文献\cite{yu2022graph}的经验研究发现使用了不同图增强方法的推荐模型性能甚至不如没有使用图增强的性能。图增强对于推荐任务的可靠性和有效性存疑,未来研究需要进一步的更深入地调查和原理性地探索。另外,未来的研究需要系统性地考虑伪正例和伪负例问题,并且充分利用物品的图像、文本信息构建更细粒度的自监督信号,使推荐模型受益于更可靠和更密集的自监督信号,是提升推荐性能的重要途径。

(2)\textit{在参数更新方式方面},以MOCO\cite{He:2020:CVPR}为代表的对比学习模型采用异步更新策略,即分别采用梯度法和动量法更新两个网络分支参数,在诸多任务上取得了出色的表现。然而,为何动量法的更新策略有效,仍是一个自监督对比学习尚需回答的问题。本文的损失函数校正虽然针对同步更新策略设计,但实质上通过校正损失函数来调节梯度,改变了标准对比损失的梯度法参数更新方式。深刻理解动量法工作机理,将基于梯度法的参数方式的改进推广至动量法,设计更优的动量更新策略以提升自监督对比学习效果,是自监督对比学习领域一个值得探索的重要理论和实践问题。

(3)\textit{在网络架构设计方面},以BYOL\cite{BYOL:2020:NIPS}为代表的对比学习模型采用异构的网络架构,分别使用了两个不对称的编码器,称为在线网络和目标网络。通过使用精致的正则化技术,在无需负样本的情况下也避免了模型坍塌(即为所有样本输出相同的特征表示),取得了出色的泛化性能。在推荐中,样本特征提取所依赖的图卷积网络不同于提取图像数据的残差网络,具有结构上的特殊性。如何设计符合推荐场景的异构网络架构,是一个值得探索的研究方向。

(4)\textit{在编码器设计方面},本文只使用了MF和LightGCN作为代表性编码器。尽管对于基于ID的推荐(IDRec),上述编码器已具备足够的表达能力(expressiveness),但表达能力并不是唯一的考量,更优的编码器设计有助于减少近似误差(Approximation Error)从而提升泛化性能\cite{sugiyama2022machine}。特别是目前预训练的模态编码器如BERT和视觉Transformer,在建模物品的原始模态特征(如文本和图像)方面变得越来越强大。最近的研究\cite{yuan2023go}显示,即使在非冷启动的推荐设置中,基于多模态编码器的推荐模型已经与最先进的基于ID的编码器取得了相当的性能,对传统的基于ID的编码器提出了挑战。如何设计更具特征通用性的物品编码器,从而使个性化推荐受益于自然语言处理和计算机视觉领域的技术进步,是一个值得探索的研究方向。

